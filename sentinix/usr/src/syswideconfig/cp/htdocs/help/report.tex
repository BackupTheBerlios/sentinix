\documentclass{article}
% Handle pdflatex nicely
\ifx\pdfoutput\undefined\else\usepackage{times}\fi
\usepackage{fancyhdr}
\pagestyle{fancy}
\fancyhead[LE,RO]{\textit{Nessus Report}}
\fancyfoot[LE,RO]{}
\pagenumbering{roman}
\title{\vspace*{100pt}\Huge Report of a Nessus scan\normalsize}
%
% You might want to change this : 
%
\author{Nessus Security Scanner}
\begin{document}
\maketitle
\newpage
\tableofcontents
\newpage
\section*{Introduction}
In this test, Nessus has tested 9 hosts and found \textbf{54 severe security holes}, as well as 113 security warnings and 303 notes.These problems can easily be used to break into your network. You should have a close look at them and correct them as soon as possible.\\
Note that there is a big number of problems for a single network of this size.\\
We strongly suggest that you correct them as soon as you can, although we know it is not always possible.\\
We recommand that you take a closer look at \verb+10.163.156.10+, as it is the host that is the most likely to be the entry point of any cracker.
On the overall, Nessus has given to the security of this network the mark E because of the number of vulnerabilities found. A script kid should be able to break into your network rather easily.\\
There is room for improvement, and \textbf{we strongly suggest that you take the appropriate measures to solve these problems \textit{as soon as possible}}
If you were considering hiring some security consultant to determine the security of your network, we strongly suggest you do so, because this should save your network.
\newpage
\section{10.163.156.10}
\subsection{Open ports (TCP and UDP)}
\verb+10.163.156.10+ has the following ports that are open : 
\begin{itemize}
\item\verb+echo (7/tcp)+
\item\verb+telnet (23/tcp)+
\item\verb+ssh (22/tcp)+
\item\verb+ftp (21/tcp)+
\item\verb+chargen (19/tcp)+
\item\verb+daytime (13/tcp)+
\item\verb+discard (9/tcp)+
\item\verb+smtp (25/tcp)+
\item\verb+time (37/tcp)+
\item\verb+finger (79/tcp)+
\item\verb+sunrpc (111/tcp)+
\item\verb+login (513/tcp)+
\item\verb+exec (512/tcp)+
\item\verb+printer (515/tcp)+
\item\verb+shell (514/tcp)+
\item\verb+uucp (540/tcp)+
\item\verb+sometimes-rpc16 (32776/udp)+
\item\verb+sometimes-rpc14 (32775/udp)+
\item\verb+sometimes-rpc10 (32773/udp)+
\item\verb+lockd (4045/udp)+
\item\verb+snmp (161/udp)+
\item\verb+sometimes-rpc22 (32779/udp)+
\item\verb+general/tcp+
\item\verb+sometimes-rpc18 (32777/udp)+
\item\verb+sometimes-rpc20 (32778/udp)+
\item\verb+dtspc (6112/tcp)+
\item\verb+sometimes-rpc13 (32775/tcp)+
\item\verb+sometimes-rpc9 (32773/tcp)+
\item\verb+sunrpc (111/udp)+
\item\verb+sometimes-rpc8 (32772/udp)+
\item\verb+sometimes-rpc5 (32771/tcp)+
\item\verb+sometimes-rpc12 (32774/udp)+
\item\verb+sometimes-rpc7 (32772/tcp)+
\item\verb+sometimes-rpc11 (32774/tcp)+
\item\verb+lockd (4045/tcp)+
\item\verb+sometimes-rpc24 (32780/udp)+
\item\verb+sometimes-rpc15 (32776/tcp)+
\item\verb+unknown (32785/udp)+
\item\verb+sometimes-rpc19 (32778/tcp)+
\item\verb+unknown (32788/udp)+
\item\verb+sometimes-rpc21 (32779/tcp)+
\item\verb+xdmcp (177/udp)+
\item\verb+font-service (7100/tcp)+
\item\verb+echo (7/udp)+
\item\verb+daytime (13/udp)+
\end{itemize}
You should disable the services that you do not use, as they are potential security flaws.
\subsection{Details of the vulnerabilities}
\subsubsection{Problems regarding : echo (7/tcp)}
Security warnings :\\
\begin{itemize}
\item \begin{verbatim}
The 'echo' port is open. This port is
not of any use nowadays, and may be a source of problems, 
since it can be used along with other ports to perform a denial
of service. You should really disable this service.

Risk factor : Low

Solution : comment out 'echo' in /etc/inetd.conf
CVE : CVE-1999-0103
\end{verbatim}\end{itemize}
Security note :\\
\begin{itemize}
\item \begin{verbatim}
An echo server is running on this port
\end{verbatim}\end{itemize}
\subsubsection{Problems regarding : telnet (23/tcp)}
Security holes :\\
\begin{itemize}
\item \begin{verbatim}
The Telnet server does not return an expected number of replies
when it receives a long sequence of 'Are You There' commands.
This probably means it overflows one of its internal buffers and
crashes. It is likely an attacker could abuse this bug to gain
control over the remote host's superuser.

For more information, see:
http://www.team-teso.net/advisories/teso-advisory-011.tar.gz

Solution: Comment out the 'telnet' line in /etc/inetd.conf.
Risk factor : High
CVE : CVE-2001-0554
BID : 3064
\end{verbatim}\item \begin{verbatim}
It is possible to reboot the remote host by connecting to the telnet
port and providing a bad username and password.
 

This vulnerability is documented as Cisco Bug ID CSCdw81244.

An attacker may use this flaw to prevent your access point from
working properly.

Solution : http://www.cisco.com/warp/public/707/Aironet-Telnet.shtml
Risk factor : High
CVE : CAN-2002-0545
BID : 4461
\end{verbatim}\end{itemize}
Security warnings :\\
\begin{itemize}
\item \begin{verbatim}
The Telnet service is running.
This service is dangerous in the sense that
it is not ciphered - that is, everyone can sniff
the data that passes between the telnet client
and the telnet server. This includes logins
and passwords.

You should disable this service and use OpenSSH instead.
(www.openssh.com)

Solution : Comment out the 'telnet' line in /etc/inetd.conf.

Risk factor : Low
CVE : CAN-1999-0619
\end{verbatim}\end{itemize}
Security note :\\
\begin{itemize}
\item \begin{verbatim}
A telnet server seems to be running on this port
\end{verbatim}\item \begin{verbatim}
Remote telnet banner :


SunOS 5.8

\end{verbatim}\end{itemize}
\subsubsection{Problems regarding : ssh (22/tcp)}
Security holes :\\
\begin{itemize}
\item \begin{verbatim}
You are running a version of OpenSSH which is older than 3.4

There is a flaw in this version that can be exploited remotely to
give an attacker a shell on this host.

Note that several distribution patched this hole without changing
the version number of OpenSSH. Since Nessus solely relied on the
banner of the remote SSH server to perform this check, this might
be a false positive.

If you are running a RedHat host, make sure that the command :
          rpm -q openssh-server
   
Returns :
 openssh-server-3.1p1-6


Solution : Upgrade to OpenSSH 3.4 or contact your vendor for a patch
Risk factor : High
CVE : CAN-2002-0639, CAN-2002-0640
BID : 5093
\end{verbatim}\item \begin{verbatim}
 
You are running a version of OpenSSH which is older than 3.0.2.

Versions prior than 3.0.2 are vulnerable to an environment
variables export that can allow a local user to execute
command with root privileges.
This problem affect only versions prior than 3.0.2, and when
the UseLogin feature is enabled (usually disabled by default)

Solution : Upgrade to OpenSSH 3.0.2 or apply the patch for prior
versions. (Available at: ftp://ftp.openbsd.org/pub/OpenBSD/OpenSSH)

Risk factor : High (If UseLogin is enabled, and locally)
CVE : CVE-2001-0872
BID : 3614
\end{verbatim}\item \begin{verbatim}
You are running a version of OpenSSH which is older than 3.0.1.

Versions older than 3.0.1 are vulnerable to a flaw in which
an attacker may authenticate, provided that Kerberos V support
has been enabled (which is not the case by default). 
It is also vulnerable as an excessive memory clearing bug, 
believed to be unexploitable.

*** You may ignore this warning if this host is not using
*** Kerberos V

Solution : Upgrade to OpenSSH 3.0.1
Risk factor : Low (if you are not using Kerberos) or High (if kerberos
 is enabled)
CVE : CVE-2002-0083
BID : 3560, 4560, 4241
\end{verbatim}\item \begin{verbatim}
You are running a version of OpenSSH older than OpenSSH 3.2.1

A buffer overflow exists in the daemon if AFS is enabled on
your system, or if the options KerberosTgtPassing or
AFSTokenPassing are enabled.  Even in this scenario, the
vulnerability may be avoided by enabling UsePrivilegeSeparation.

Versions prior to 2.9.9 are vulnerable to a remote root
exploit. Versions prior to 3.2.1 are vulnerable to a local
root exploit.

Solution :
Upgrade to the latest version of OpenSSH

Risk factor : High
CVE : CAN-2002-0575
BID : 4560
\end{verbatim}\end{itemize}
Security warnings :\\
\begin{itemize}
\item \begin{verbatim}
The remote SSH daemon supports connections made
using the version 1.33 and/or 1.5 of the SSH protocol.

These protocols are not completely cryptographically
safe so they should not be used.

Solution : 
 If you use OpenSSH, set the option 'Protocol' to '2'
 If you use SSH.com's set the option 'Ssh1Compatibility' to 'no'
  
Risk factor : Low
\end{verbatim}\end{itemize}
Security note :\\
\begin{itemize}
\item \begin{verbatim}
An ssh server is running on this port
\end{verbatim}\item \begin{verbatim}
The remote SSH daemon supports the following versions of the
SSH protocol :

  . 1.33
  . 1.5
  . 1.99
  . 2.0

\end{verbatim}\item \begin{verbatim}
Remote SSH version : SSH-1.99-OpenSSH_2.3.0p1
\end{verbatim}\end{itemize}
\subsubsection{Problems regarding : ftp (21/tcp)}
Security holes :\\
\begin{itemize}
\item \begin{verbatim}
The remote FTP server seems to be vulnerable to an exhaustion
attack which may makes it consume all available memory on the remote
host when it receives the command :

 NLST /../*/../*/../*/../*/../*/../*/../*/../*/../*/../ 
 

Solution : upgrade to ProFTPd 1.2.2 if the remote server is proftpd,
or contact your vendor for a patch.

Reference : http://online.securityfocus.com/archive/1/169069

Risk factor : High
\end{verbatim}\item \begin{verbatim}
You seem to be running an FTP server which is vulnerable to the 'glob
 heap corruption'
flaw, which is known to be exploitable remotely against this server.
 An attacker may use 
this flaw to execute arbitrary commands on this host.

Solution: Upgrade your ftp server software to the latest version.
Risk factor : High

CVE : CVE-2001-0550
BID : 3581
\end{verbatim}\end{itemize}
Security warnings :\\
\begin{itemize}
\item \begin{verbatim}
This FTP service allows anonymous logins. If you do not
 want to share data with anyone you do not know, then you should
 deactivate
 the anonymous account, since it can only cause troubles.
 Under most Unix system, doing : 
  echo ftp >> /etc/ftpusers
 will correct this.
 
 Risk factor : Low
CVE : CAN-1999-0497
\end{verbatim}\item \begin{verbatim}
It is possible to gather the
real path of the public area of the ftp server
(like /home/ftp) by issuing the following
command :

 CWD
 
This problem may help an attacker to find where
to put a .rhost file using other security
flaws.

Risk factor : Low
CVE : CVE-1999-0201
\end{verbatim}\item \begin{verbatim}
The remote FTP server allows users to make any amount
of PASV commands, thus blocking the free ports for legitimate services
 and
consuming file descriptors.

Solution: upgrade your FTP server to a version which solves this
 problem.

Risk factor : Medium
CVE : CVE-1999-0079
BID : 271
\end{verbatim}\end{itemize}
Security note :\\
\begin{itemize}
\item \begin{verbatim}
An FTP server is running on this port.
Here is its banner : 
220 unknown FTP server (SunOS 5.8) ready.
\end{verbatim}\item \begin{verbatim}
Remote FTP server banner :
220 unknown FTP server (SunOS 5.8) ready.
\end{verbatim}\end{itemize}
\subsubsection{Problems regarding : chargen (19/tcp)}
Security warnings :\\
\begin{itemize}
\item \begin{verbatim}
The chargen service is running.
The 'chargen' service should only be enabled when testing the machine.
 

When contacted, chargen responds with some random characters
 (something
like all the characters in the alphabet in a row). When contacted via
 UDP, it 
will respond with a single UDP packet. When contacted via TCP, it will
 
continue spewing characters until the client closes the connection. 

An easy attack is 'pingpong' in which an attacker spoofs a packet
 between two
machines running chargen. This will cause them to spew characters at
 each 
other, slowing the machines down and saturating the network.
      
Solution : disable this service in /etc/inetd.conf.

Risk factor : Low
CVE : CVE-1999-0103
\end{verbatim}\end{itemize}
Security note :\\
\begin{itemize}
\item \begin{verbatim}
Chargen is running on this port
\end{verbatim}\end{itemize}
\subsubsection{Problems regarding : daytime (13/tcp)}
Security warnings :\\
\begin{itemize}
\item \begin{verbatim}
The daytime service is running.
The date format issued by this service
may sometimes help an attacker to guess
the operating system type. 

In addition to that, when the UDP version of
daytime is running, an attacker may link it 
to the echo port using spoofing, thus creating
a possible denial of service.

Solution : disable this service in /etc/inetd.conf.

Risk factor : Low
CVE : CVE-1999-0103
\end{verbatim}\end{itemize}
\subsubsection{Problems regarding : smtp (25/tcp)}
Security holes :\\
\begin{itemize}
\item \begin{verbatim}
The remote sendmail server, according to its version number,
may be vulnerable to the -bt overflow attack which
allows any local user to execute arbitrary commands as root.

Solution : upgrade to the latest version of Sendmail
Risk factor : High
Note : This vulnerability is _local_ only
\end{verbatim}\item \begin{verbatim}
The remote sendmail server, according to its version number,
may be vulnerable to a buffer overflow its DNS handling code.

The owner of a malicious name server could use this flaw
to execute arbitrary code on this host.


Solution : Upgrade to Sendmail 8.12.5
Risk factor : High
CVE : CAN-2002-0906
BID : 5122
\end{verbatim}\end{itemize}
Security warnings :\\
\begin{itemize}
\item \begin{verbatim}
The remote SMTP server
answers to the EXPN and/or VRFY commands.

The EXPN command can be used to find 
the delivery address of mail aliases, or 
even the full name of the recipients, and 
the VRFY command may be used to check the 
validity of an account.


Your mailer should not allow remote users to
use any of these commands, because it gives
them too much information.


Solution : if you are using Sendmail, add the 
option
 O PrivacyOptions=goaway
in /etc/sendmail.cf.

Risk factor : Low
CVE : CAN-1999-0531
\end{verbatim}\item \begin{verbatim}
The remote SMTP server is vulnerable to a redirection
attack. That is, if a mail is sent to :

  user@hostname1@victim
  
Then the remote SMTP server (victim) will happily send the
mail to :
  user@hostname1
  
Using this flaw, an attacker may route a message
through your firewall, in order to exploit other
SMTP servers that can not be reached from the
outside.

*** THIS WARNING MAY BE A FALSE POSITIVE, SINCE
    SOME SMTP SERVERS LIKE POSTFIX WILL NOT
    COMPLAIN BUT DROP THIS MESSAGE ***
    
    
Solution : if you are using sendmail, then at the top
of ruleset 98, in /etc/sendmail.cf, insert :
R$*@$*@$*       $#error $@ 5.7.1 $: '551 Sorry, no redirections.'

Risk factor : Low
\end{verbatim}\item \begin{verbatim}
The remote SMTP server allows the relaying. This means that
it allows spammers to use your mail server to send their mails to
the world, thus wasting your network bandwidth.

Risk factor : Low/Medium

Solution : configure your SMTP server so that it can't be used as a
 relay
           any more.
CVE : CAN-1999-0512
\end{verbatim}\item \begin{verbatim}
The remote SMTP server allows anyone to
use it as a mail relay, provided that the source address 
is set to '<>'. 
This problem allows any spammer to use your mail server 
to spam the world, thus blacklisting your mailserver, and
using your network resources.

Risk factor : Medium

Solution : reconfigure this server properly
CVE : CVE-1999-0819
\end{verbatim}\item \begin{verbatim}
The remote sendmail server, according to its version number,
might be vulnerable to a queue destruction when a local user
runs
 sendmail -q -h1000

If you system does not allow users to process the queue (which
is the default), you are not vulnerable.

Solution : upgrade to the latest version of Sendmail or 
do not allow users to process the queue (RestrictQRun option)
Risk factor : Low
Note : This vulnerability is _local_ only
CVE : CAN-2001-0714
BID : 3378
\end{verbatim}\item \begin{verbatim}
According to the version number of the remote mail server, 
a local user may be able to obtain the complete mail configuration
and other interesting information about the mail queue even if
he is not allowed to access those information directly, by running
 sendmail -q -d0-nnnn.xxx
where nnnn & xxx are debugging levels.

If users are not allowed to process the queue (which is the default)
then you are not vulnerable.

Solution : upgrade to the latest version of Sendmail or 
do not allow users to process the queue (RestrictQRun option)
Risk factor : Very low / none
Note : This vulnerability is _local_ only
CVE : CAN-2001-0715
BID : 3898
\end{verbatim}\end{itemize}
Security note :\\
\begin{itemize}
\item \begin{verbatim}
An SMTP server is running on this port
Here is its banner : 
220 sparky.fr.nessus.org ESMTP Sendmail 8.9.3+Sun/8.9.3; Fri, 21 Feb
 2003 15:53:28 GMT
\end{verbatim}\item \begin{verbatim}
Remote SMTP server banner :
220 sparky.fr.nessus.org ESMTP Sendmail 8.9.3+Sun/8.9.3; Fri, 21 Feb
 2003 15:54:20 GMT

\end{verbatim}\end{itemize}
\subsubsection{Problems regarding : time (37/tcp)}
Security note :\\
\begin{itemize}
\item \begin{verbatim}
A time server seems to be running on this port
\end{verbatim}\end{itemize}
\subsubsection{Problems regarding : finger (79/tcp)}
Security warnings :\\
\begin{itemize}
\item \begin{verbatim}
The 'finger' service provides useful information
to attackers, since it allow them to gain usernames, check if a
 machine
is being used, and so on... 

Risk factor : Low

Solution : comment out the 'finger' line in /etc/inetd.conf
CVE : CVE-1999-0612
\end{verbatim}\item \begin{verbatim}
The remote finger daemon accepts
to redirect requests. That is, users can perform
requests like :
  finger user@host@victim

This allows an attacker to use your computer
as a relay to gather information on another
network, making the other network think you
are making the requests.

Solution: disable your finger daemon (comment out
the finger line in /etc/inetd.conf) or 
install a more secure one.

Risk factor : Low
CVE : CAN-1999-0105
\end{verbatim}\item \begin{verbatim}
There is a bug in the finger service
which will make it display the list of the accounts that
have never been used, when anyone issues the request :

  finger 'a b c d e f g h'@target
  
This list will help an attacker to guess the operating
system type. It will also tell him which accounts have
never been used, which will often make him focus his
attacks on these accounts.

Solution : disable the finger service in /etc/inetd.conf, or
apply the patches from Sun.

Risk factor : Medium
BID : 3457
\end{verbatim}\end{itemize}
Security note :\\
\begin{itemize}
\item \begin{verbatim}
A finger server seems to be running on this port
\end{verbatim}\end{itemize}
\subsubsection{Problems regarding : sunrpc (111/tcp)}
Security note :\\
\begin{itemize}
\item \begin{verbatim}
RPC program #100000 version 4 'portmapper' (portmap sunrpc rpcbind) is
 running on this port
\end{verbatim}\item \begin{verbatim}
RPC program #100000 version 3 'portmapper' (portmap sunrpc rpcbind) is
 running on this port
\end{verbatim}\item \begin{verbatim}
RPC program #100000 version 2 'portmapper' (portmap sunrpc rpcbind) is
 running on this port
\end{verbatim}\end{itemize}
\subsubsection{Problems regarding : login (513/tcp)}
Security warnings :\\
\begin{itemize}
\item \begin{verbatim}
The rlogin service is running.
This service is dangerous in the sense that
it is not ciphered - that is, everyone can sniff
the data that passes between the rlogin client
and the rlogin server. This includes logins
and passwords.

You should disable this service and use openssh instead
(www.openssh.com)

Solution : Comment out the 'rlogin' line in /etc/inetd.conf.

Risk factor : Low
CVE : CAN-1999-0651
\end{verbatim}\end{itemize}
\subsubsection{Problems regarding : exec (512/tcp)}
Security warnings :\\
\begin{itemize}
\item \begin{verbatim}
The rexecd service is open. 
Because rexecd does not provide any good
means of authentication, it can be
used by an attacker to scan a third party
host, giving you troubles or bypassing
your firewall.

Solution : comment out the 'exec' line 
in /etc/inetd.conf.

Risk factor : Medium
CVE : CAN-1999-0618
\end{verbatim}\end{itemize}
\subsubsection{Problems regarding : printer (515/tcp)}
Security note :\\
\begin{itemize}
\item \begin{verbatim}
A LPD server seems to be running on this port
\end{verbatim}\end{itemize}
\subsubsection{Problems regarding : shell (514/tcp)}
Security warnings :\\
\begin{itemize}
\item \begin{verbatim}
The rsh service is running.
This service is dangerous in the sense that
it is not ciphered - that is, everyone can sniff
the data that passes between the rsh client
and the rsh server. This includes logins
and passwords.

You should disable this service and use ssh instead.

Solution : Comment out the 'rsh' line in /etc/inetd.conf.

Risk factor : Low
CVE : CAN-1999-0651
\end{verbatim}\end{itemize}
\subsubsection{Problems regarding : uucp (540/tcp)}
Security note :\\
\begin{itemize}
\item \begin{verbatim}
An UUCP server seems to be running on this port
\end{verbatim}\end{itemize}
\subsubsection{Problems regarding : sometimes-rpc16 (32776/udp)}
Security warnings :\\
\begin{itemize}
\item \begin{verbatim}
The sprayd RPC service is running. 
If you do not use this service, then
disable it as it may become a security
threat in the future, if a vulnerability
is discovered.

Risk factor : Low
CVE : CAN-1999-0613
\end{verbatim}\end{itemize}
Security note :\\
\begin{itemize}
\item \begin{verbatim}
RPC program #100012 version 1 'sprayd' (spray) is running on this port
\end{verbatim}\end{itemize}
\subsubsection{Problems regarding : sometimes-rpc14 (32775/udp)}
Security warnings :\\
\begin{itemize}
\item \begin{verbatim}
The rusersd RPC service is running. 
It provides an attacker interesting
information such as how often the
system is being used, the names of
the users, and so on.
 
It usually not a good idea to leave this
service open.


Risk factor : Low
CVE : CVE-1999-0626
\end{verbatim}\end{itemize}
Security note :\\
\begin{itemize}
\item \begin{verbatim}
RPC program #100002 version 2 'rusersd' (rusers) is running on this
 port
\end{verbatim}\item \begin{verbatim}
RPC program #100002 version 3 'rusersd' (rusers) is running on this
 port
\end{verbatim}\end{itemize}
\subsubsection{Problems regarding : sometimes-rpc10 (32773/udp)}
Security holes :\\
\begin{itemize}
\item \begin{verbatim}
The sadmin RPC service is running. 
There is a bug in Solaris versions of
this service that allow an intruder to
execute arbitrary commands on your system.

Solution : disable this service
Risk factor : High
CVE : CVE-1999-0977
BID : 866
\end{verbatim}\end{itemize}
Security note :\\
\begin{itemize}
\item \begin{verbatim}
RPC program #100232 version 10 'sadmind' is running on this port
\end{verbatim}\end{itemize}
\subsubsection{Problems regarding : lockd (4045/udp)}
Security warnings :\\
\begin{itemize}
\item \begin{verbatim}
The nlockmgr RPC service is running. 
If you do not use this service, then
disable it as it may become a security
threat in the future, if a vulnerability
is discovered.

Risk factor : Low
CVE : CVE-2000-0508
BID : 1372
\end{verbatim}\end{itemize}
Security note :\\
\begin{itemize}
\item \begin{verbatim}
RPC program #100021 version 1 'nlockmgr' is running on this port
\end{verbatim}\item \begin{verbatim}
RPC program #100021 version 2 'nlockmgr' is running on this port
\end{verbatim}\item \begin{verbatim}
RPC program #100021 version 3 'nlockmgr' is running on this port
\end{verbatim}\item \begin{verbatim}
RPC program #100021 version 4 'nlockmgr' is running on this port
\end{verbatim}\end{itemize}
\subsubsection{Problems regarding : snmp (161/udp)}
Security holes :\\
\begin{itemize}
\item \begin{verbatim}
The device answered to more than 4 community strings.
This may be a false positive or a community-less SNMP server
HP printers answer to all community strings.

SNMP Agent responded as expected with community name: public
SNMP Agent responded as expected with community name: private
SNMP Agent responded as expected with community name: write
SNMP Agent responded as expected with community name: all
SNMP Agent responded as expected with community name: monitor
SNMP Agent responded as expected with community name: agent
SNMP Agent responded as expected with community name: manager
SNMP Agent responded as expected with community name: OrigEquipMfr
SNMP Agent responded as expected with community name: admin
SNMP Agent responded as expected with community name: default
SNMP Agent responded as expected with community name: password
SNMP Agent responded as expected with community name: tivoli
SNMP Agent responded as expected with community name: openview
SNMP Agent responded as expected with community name: community
SNMP Agent responded as expected with community name: snmp
SNMP Agent responded as expected with community name: snmpd
SNMP Agent responded as expected with community name: Secret C0de
SNMP Agent responded as expected with community name: security
SNMP Agent responded as expected with community name: all private
SNMP Agent responded as expected with community name: rmon
SNMP Agent responded as expected with community name: rmon_admin
SNMP Agent responded as expected with community name: hp_admin
SNMP Agent responded as expected with community name: NoGaH$@!
SNMP Agent responded as expected with community name: 0392a0
SNMP Agent responded as expected with community name: xyzzy
SNMP Agent responded as expected with community name: agent_steal
SNMP Agent responded as expected with community name: freekevin
SNMP Agent responded as expected with community name: fubar
SNMP Agent responded as expected with community name: secret
SNMP Agent responded as expected with community name: cisco
SNMP Agent responded as expected with community name: apc
SNMP Agent responded as expected with community name: ANYCOM
SNMP Agent responded as expected with community name: cable-docsis
SNMP Agent responded as expected with community name: c
SNMP Agent responded as expected with community name: cc
SNMP Agent responded as expected with community name: Cisco router
SNMP Agent responded as expected with community name: cascade
SNMP Agent responded as expected with community name: comcomcom
CVE : CAN-1999-0186
BID : 177
\end{verbatim}\item \begin{verbatim}
It was possible to disable the remote SNMP daemon by sending
a malformed packet advertising bogus length fields.

An attacker may use this flaw to prevent you from using
SNMP to administer your network (or use other flaws
to execute arbitrary code with the privileges of the 
SNMP daemon)

Solution : see www.cert.org/advisories/CA-2002-03.html
Risk factor : High
CVE : CAN-2002-0013
\end{verbatim}\end{itemize}
Security warnings :\\
\begin{itemize}
\item \begin{verbatim}
A SNMP server is running on this host
\end{verbatim}\end{itemize}
Security note :\\
\begin{itemize}
\item \begin{verbatim}
Using SNMP, we could determine that the remote operating system is :
Sun SNMP Agent, Ultra-1
\end{verbatim}\end{itemize}
\subsubsection{Problems regarding : sometimes-rpc22 (32779/udp)}
Security holes :\\
\begin{itemize}
\item \begin{verbatim}
The cmsd RPC service is running. 
This service has a long history of 
security holes, so you should really
know what you are doing if you decide
to let it run.

* NO SECURITY HOLE REGARDING THIS
  PROGRAM HAS BEEN TESTED, SO
  THIS MIGHT BE A FALSE POSITIVE *

We suggest that you disable this
service.


Risk factor : High
CVE : CVE-1999-0320, CVE-1999-0696
BID : 428
\end{verbatim}\end{itemize}
Security note :\\
\begin{itemize}
\item \begin{verbatim}
RPC program #100068 version 2 is running on this port
\end{verbatim}\item \begin{verbatim}
RPC program #100068 version 3 is running on this port
\end{verbatim}\item \begin{verbatim}
RPC program #100068 version 4 is running on this port
\end{verbatim}\item \begin{verbatim}
RPC program #100068 version 5 is running on this port
\end{verbatim}\end{itemize}
\subsubsection{Problems regarding : general/tcp}
Security note :\\
\begin{itemize}
\item \begin{verbatim}
QueSO has found out that the remote host OS is 
* Standard: Solaris 2.x, Linux 2.1.???, Linux 2.2, MacOS


CVE : CAN-1999-0454
\end{verbatim}\end{itemize}
\subsubsection{Problems regarding : sometimes-rpc18 (32777/udp)}
Security holes :\\
\begin{itemize}
\item \begin{verbatim}
The rpc.walld RPC service is running. 
Some versions of this server allow an attacker to gain
root access remotely, by consuming the resources of the 
remote host then sending a specially formed packet with
format strings to this host.

Solaris 2.5.1, 2.6, 7 and 8 are vulnerable to this
issue. Other operating systems might be affected as well.

*** Nessus did not check for this vulnerability, 
*** so this might be a false positive

Solution : Deactivate this service.
Risk factor : High
CVE : CAN-2002-0573
BID : 4639
\end{verbatim}\end{itemize}
Security warnings :\\
\begin{itemize}
\item \begin{verbatim}
The walld RPC service is running. 
It is usually used by the administrator
to tell something to the users of a
network by making a message appear
on their screen.

Since this service lacks any kind
of authentication, an attacker
may use it to trick users into
doing something (change their password,
leave the console, or worse), by sending
a message which would appear to be
written by the administrator.

It can also be used as a denial of service
attack, by continually sending garbage
to the users screens, preventing them
from working properly.

Solution : Deactivate this service.

Risk factor : Medium
CVE : CVE-1999-0181
\end{verbatim}\end{itemize}
Security note :\\
\begin{itemize}
\item \begin{verbatim}
RPC program #100008 version 1 'walld' (rwall shutdown) is running on
 this port
\end{verbatim}\end{itemize}
\subsubsection{Problems regarding : sometimes-rpc20 (32778/udp)}
Security warnings :\\
\begin{itemize}
\item \begin{verbatim}
The rstatd RPC service is running. 
It provides an attacker interesting
information such as :

 - the CPU usage
 - the system uptime
 - its network usage
 - and more
 
Usually, it is not a good idea to let this
service open


Risk factor : Low
CVE : CAN-1999-0624
\end{verbatim}\end{itemize}
Security note :\\
\begin{itemize}
\item \begin{verbatim}
RPC program #100001 version 2 'rstatd' (rstat rup perfmeter rstat_svc)
 is running on this port
\end{verbatim}\item \begin{verbatim}
RPC program #100001 version 3 'rstatd' (rstat rup perfmeter rstat_svc)
 is running on this port
\end{verbatim}\item \begin{verbatim}
RPC program #100001 version 4 'rstatd' (rstat rup perfmeter rstat_svc)
 is running on this port
\end{verbatim}\end{itemize}
\subsubsection{Problems regarding : dtspc (6112/tcp)}
Security holes :\\
\begin{itemize}
\item \begin{verbatim}
The 'dtspcd' service is running.

Some versions of this daemon are vulnerable to
a buffer overflow attack which allows an attacker
to gain root privileges

*** This warning might be a false positive,
*** as no real overflow was performed

Solution : See http://www.cert.org/advisories/CA-2001-31.html
to determine if you are vulnerable or deactivate
this service (comment out the line 'dtspc' in /etc/inetd.conf)

Risk factor : High
CVE : CVE-2001-0803
BID : 3517
\end{verbatim}\end{itemize}
\subsubsection{Problems regarding : sometimes-rpc13 (32775/tcp)}
Security holes :\\
\begin{itemize}
\item \begin{verbatim}
The cachefsd RPC service is running. 
Some versions of this server allow an attacker to gain
root access remotely, by consuming the resources of the 
remote host then sending a specially formed packet with
format strings to this host.

Solaris 2.5.1, 2.6, 7 and 8 are vulnerable to this
issue. Other operating systems might be affected as well.

*** Nessus did not check for this vulnerability, 
*** so this might be a false positive

Solution : Deactivate this service - there is no patch at this time
    /etc/init.d/cachefs.daemon stop
Risk factor : High
CVE : CAN-2002-0084, CAN-2002-0033
BID : 4631
\end{verbatim}\end{itemize}
Security note :\\
\begin{itemize}
\item \begin{verbatim}
RPC program #100235 version 1 is running on this port
\end{verbatim}\end{itemize}
\subsubsection{Problems regarding : sometimes-rpc9 (32773/tcp)}
Security holes :\\
\begin{itemize}
\item \begin{verbatim}
The tooltalk RPC service is running.

There is a format string bug in many versions
of this service, which allow an attacker to gain
root remotely.

In addition to this, several versions of this service
allow remote attackers to overwrite abitrary memory
locations with a zero and possibly gain privileges
via a file descriptor argument in an AUTH_UNIX 
procedure call which is used as a table index by the
_TT_ISCLOSE procedure.

*** This warning may be a false positive since the presence
*** of the bug was not verified locally.
    
Solution : Disable this service or patch it
See also : CERT Advisories CA-2001-27 and CA-2002-20

Risk factor : High
CVE : CAN-2002-0677, CVE-2001-0717, CVE-2002-0679
BID : 3382
\end{verbatim}\end{itemize}
Security note :\\
\begin{itemize}
\item \begin{verbatim}
RPC program #100083 version 1 is running on this port
\end{verbatim}\end{itemize}
\subsubsection{Problems regarding : sunrpc (111/udp)}
Security note :\\
\begin{itemize}
\item \begin{verbatim}
RPC program #100000 version 4 'portmapper' (portmap sunrpc rpcbind) is
 running on this port
\end{verbatim}\item \begin{verbatim}
RPC program #100000 version 3 'portmapper' (portmap sunrpc rpcbind) is
 running on this port
\end{verbatim}\item \begin{verbatim}
RPC program #100000 version 2 'portmapper' (portmap sunrpc rpcbind) is
 running on this port
\end{verbatim}\end{itemize}
\subsubsection{Problems regarding : sometimes-rpc8 (32772/udp)}
Security note :\\
\begin{itemize}
\item \begin{verbatim}
RPC program #100300 version 3 'nisd' (rpc.nisd) is running on this
 port
\end{verbatim}\end{itemize}
\subsubsection{Problems regarding : sometimes-rpc5 (32771/tcp)}
Security note :\\
\begin{itemize}
\item \begin{verbatim}
RPC program #100300 version 3 'nisd' (rpc.nisd) is running on this
 port
\end{verbatim}\end{itemize}
\subsubsection{Problems regarding : sometimes-rpc12 (32774/udp)}
Security warnings :\\
\begin{itemize}
\item \begin{verbatim}
The rquotad RPC service is running. 
If you do not use this service, then
disable it as it may become a security
threat in the future, if a vulnerability
is discovered.

Risk factor : Low
CVE : CAN-1999-0625
\end{verbatim}\end{itemize}
Security note :\\
\begin{itemize}
\item \begin{verbatim}
RPC program #100011 version 1 'rquotad' (rquotaprog quota rquota) is
 running on this port
\end{verbatim}\end{itemize}
\subsubsection{Problems regarding : sometimes-rpc7 (32772/tcp)}
Security note :\\
\begin{itemize}
\item \begin{verbatim}
RPC program #100002 version 2 'rusersd' (rusers) is running on this
 port
\end{verbatim}\item \begin{verbatim}
RPC program #100002 version 3 'rusersd' (rusers) is running on this
 port
\end{verbatim}\end{itemize}
\subsubsection{Problems regarding : sometimes-rpc11 (32774/tcp)}
Security note :\\
\begin{itemize}
\item \begin{verbatim}
RPC program #100221 version 1 is running on this port
\end{verbatim}\end{itemize}
\subsubsection{Problems regarding : lockd (4045/tcp)}
Security note :\\
\begin{itemize}
\item \begin{verbatim}
RPC program #100021 version 1 'nlockmgr' is running on this port
\end{verbatim}\item \begin{verbatim}
RPC program #100021 version 2 'nlockmgr' is running on this port
\end{verbatim}\item \begin{verbatim}
RPC program #100021 version 3 'nlockmgr' is running on this port
\end{verbatim}\item \begin{verbatim}
RPC program #100021 version 4 'nlockmgr' is running on this port
\end{verbatim}\end{itemize}
\subsubsection{Problems regarding : sometimes-rpc24 (32780/udp)}
Security warnings :\\
\begin{itemize}
\item \begin{verbatim}
The statd RPC service is running. 
This service has a long history of 
security holes, so you should really
know what you are doing if you decide
to let it run.

* NO SECURITY HOLES REGARDING THIS
  PROGRAM HAVE BEEN TESTED, SO
  THIS MIGHT BE A FALSE POSITIVE *

We suggest that you disable this
service.


Risk factor : High
CVE : CVE-1999-0493
BID : 450
\end{verbatim}\end{itemize}
Security note :\\
\begin{itemize}
\item \begin{verbatim}
RPC program #100024 version 1 'status' is running on this port
\end{verbatim}\item \begin{verbatim}
RPC program #100133 version 1 is running on this port
\end{verbatim}\end{itemize}
\subsubsection{Problems regarding : sometimes-rpc15 (32776/tcp)}
Security note :\\
\begin{itemize}
\item \begin{verbatim}
RPC program #100024 version 1 'status' is running on this port
\end{verbatim}\item \begin{verbatim}
RPC program #100133 version 1 is running on this port
\end{verbatim}\end{itemize}
\subsubsection{Problems regarding : unknown (32785/udp)}
Security note :\\
\begin{itemize}
\item \begin{verbatim}
RPC program #100249 version 1 is running on this port
\end{verbatim}\end{itemize}
\subsubsection{Problems regarding : sometimes-rpc19 (32778/tcp)}
Security holes :\\
\begin{itemize}
\item \begin{verbatim}
The remote RPC service 100249 (snmpXdmid) is vulnerable
to a heap overflow which allows any user to obtain a root
shell on this host.

Solution : disable this service (/etc/init.d/init.dmi stop) if you
 don't use
it, or contact Sun for a patch
Risk factor : High
CVE : CVE-2001-0236
BID : 2417
\end{verbatim}\end{itemize}
Security note :\\
\begin{itemize}
\item \begin{verbatim}
RPC program #100249 version 1 is running on this port
\end{verbatim}\end{itemize}
\subsubsection{Problems regarding : unknown (32788/udp)}
Security note :\\
\begin{itemize}
\item \begin{verbatim}
RPC program #300598 version 1 is running on this port
\end{verbatim}\item \begin{verbatim}
RPC program #805306368 version 1 is running on this port
\end{verbatim}\end{itemize}
\subsubsection{Problems regarding : sometimes-rpc21 (32779/tcp)}
Security note :\\
\begin{itemize}
\item \begin{verbatim}
RPC program #300598 version 1 is running on this port
\end{verbatim}\item \begin{verbatim}
RPC program #805306368 version 1 is running on this port
\end{verbatim}\end{itemize}
\subsubsection{Problems regarding : xdmcp (177/udp)}
Security warnings :\\
\begin{itemize}
\item \begin{verbatim}
The remote host is running XDMCP.

This protocol is used to provide X display connections for 
X terminals. XDMCP is completely insecure, since the traffic and
passwords are not encrypted. 

An attacker may use this flaw to capture all the keystrokes of
the users using this host through their X terminal, including
passwords.

Risk factor : Medium
Solution : Disable XDMCP
\end{verbatim}\end{itemize}
\subsubsection{Problems regarding : font-service (7100/tcp)}
Security holes :\\
\begin{itemize}
\item \begin{verbatim}
The remote X Font Service (xfs) might be vulnerable to a buffer
overflow.

An attacker may use this flaw to gain root on this host
remotely.

*** Note that Nessus did not actually check for the flaw
*** as details about this vulnerability are still unknown

Solution : See CERT Advisory CA-2002-34
Risk factor : High
CVE : CAN-2002-1317
\end{verbatim}\end{itemize}
\subsubsection{Problems regarding : echo (7/udp)}
Security warnings :\\
\begin{itemize}
\item \begin{verbatim}
The 'echo' port is open. This port is
not of any use nowadays, and may be a source of problems, 
since it can be used along with other ports to perform a denial
of service. You should really disable this service.

Risk factor : Low

Solution : comment out 'echo' in /etc/inetd.conf
CVE : CVE-1999-0103
\end{verbatim}\end{itemize}
\subsubsection{Problems regarding : daytime (13/udp)}
Security warnings :\\
\begin{itemize}
\item \begin{verbatim}
The daytime service is running.
The date format issued by this service
may sometimes help an attacker to guess
the operating system type. 

In addition to that, when the UDP version of
daytime is running, an attacker may link it 
to the echo port using spoofing, thus creating
a possible denial of service.

Solution : disable this service in /etc/inetd.conf.

Risk factor : Low
CVE : CVE-1999-0103
\end{verbatim}\end{itemize}
\newpage
\section{10.163.156.9}
\subsection{Open ports (TCP and UDP)}
\verb+10.163.156.9+ has the following ports that are open : 
\begin{itemize}
\item\verb+smtp (25/tcp)+
\item\verb+ftp (21/tcp)+
\item\verb+chargen (19/tcp)+
\item\verb+qotd (17/tcp)+
\item\verb+daytime (13/tcp)+
\item\verb+discard (9/tcp)+
\item\verb+echo (7/tcp)+
\item\verb+nameserver (42/tcp)+
\item\verb+http (80/tcp)+
\item\verb+nntp (119/tcp)+
\item\verb+loc-srv (135/tcp)+
\item\verb+netbios-ssn (139/tcp)+
\item\verb+microsoft-ds (445/tcp)+
\item\verb+https (443/tcp)+
\item\verb+printer (515/tcp)+
\item\verb+afpovertcp (548/tcp)+
\item\verb+nntps (563/tcp)+
\item\verb+blackjack (1025/tcp)+
\item\verb+unknown (1028/tcp)+
\item\verb+unknown (1035/tcp)+
\item\verb+netinfo (1033/tcp)+
\item\verb+iad2 (1031/tcp)+
\item\verb+ms-sql-s (1433/tcp)+
\item\verb+ms-sql-m (1434/udp)+
\item\verb+general/tcp+
\item\verb+general/udp+
\item\verb+snmp (161/udp)+
\item\verb+netbios-ns (137/udp)+
\item\verb+echo (7/udp)+
\item\verb+ms-term-serv (3389/tcp)+
\item\verb+daytime (13/udp)+
\item\verb+qotd (17/udp)+
\item\verb+iad1 (1030/udp)+
\item\verb+chargen (19/udp)+
\item\verb+iad3 (1032/udp)+
\end{itemize}
You should disable the services that you do not use, as they are potential security flaws.
\subsection{Details of the vulnerabilities}
\subsubsection{Problems regarding : smtp (25/tcp)}
Security holes :\\
\begin{itemize}
\item \begin{verbatim}
The remote SMTP server did not complain when issued the
command :
 MAIL FROM: root@this_host
 RCPT TO: /tmp/nessus_test
 
This probably means that it is possible to send mail directly
to files, which is a serious threat, since this allows
anyone to overwrite any file on the remote server.

*** This security hole might be a false positive, since
*** some MTAs will not complain to this test, but instead
*** just drop the message silently.
*** Check for the presence of file 'nessus_test' in /tmp !
   
Solution : upgrade your MTA or change it.

Risk factor : High
\end{verbatim}\item \begin{verbatim}
The remote SMTP server did not complain when issued the
command :
 MAIL FROM: |testing
 
This probably means that it is possible to send mail 
that will be bounced to a program, which is 
a serious threat, since this allows anyone to execute 
arbitrary commands on this host.

*** This security hole might be a false positive, since
*** some MTAs will not complain to this test, but instead
*** just drop the message silently
   
Solution : upgrade your MTA or change it.

Risk factor : High
CVE : CVE-1999-0203
BID : 2308
\end{verbatim}\item \begin{verbatim}
The remote SMTP server did not complain when issued the
command :
 MAIL FROM: root@this_host
 RCPT TO: |testing
 
This probably means that it is possible to send mail directly
to programs, which is a serious threat, since this allows
anyone to execute arbitrary commands on this host.

*** This security hole might be a false positive, since
*** some MTAs will not complain to this test, but instead
*** just drop the message silently.
   
Solution : upgrade your MTA or change it.

Risk factor : High
CVE : CAN-1999-0163
\end{verbatim}\end{itemize}
Security note :\\
\begin{itemize}
\item \begin{verbatim}
An SMTP server is running on this port
Here is its banner : 
220 gabbo Microsoft ESMTP MAIL Service, Version: 5.0.2195.5329 ready
 at  Fri, 21 Feb 2003 15:45:19 -0800 
\end{verbatim}\item \begin{verbatim}
Remote SMTP server banner :
220 gabbo Microsoft ESMTP MAIL Service, Version: 5.0.2195.5329 ready
 at  Fri, 21 Feb 2003 15:48:26 -0800 


\end{verbatim}\item \begin{verbatim}
For some reason, we could not send the EICAR test string to this MTA
\end{verbatim}\end{itemize}
\subsubsection{Problems regarding : ftp (21/tcp)}
Security holes :\\
\begin{itemize}
\item \begin{verbatim}
The remote FTP server closes
the connection when one of the commands is given
a too long argument. 

This probably due to a buffer overflow, which
allows anyone to execute arbitrary code
on the remote host.

This problem is threatening, because
the attackers don't need an account 
to exploit this flaw.

Solution : Upgrade your FTP server or change it
Risk factor : High
CVE : CAN-2000-0133
BID : 961
\end{verbatim}\end{itemize}
Security warnings :\\
\begin{itemize}
\item \begin{verbatim}
This FTP service allows anonymous logins. If you do not
 want to share data with anyone you do not know, then you should
 deactivate
 the anonymous account, since it can only cause troubles.
 Under most Unix system, doing : 
  echo ftp >> /etc/ftpusers
 will correct this.
 
 Risk factor : Low
CVE : CAN-1999-0497
\end{verbatim}\end{itemize}
Security note :\\
\begin{itemize}
\item \begin{verbatim}
An FTP server is running on this port.
Here is its banner : 
220 gabbo Microsoft FTP Service (Version 5.0).
\end{verbatim}\item \begin{verbatim}
Remote FTP server banner :
220 gabbo Microsoft FTP Service (Version 5.0).
\end{verbatim}\end{itemize}
\subsubsection{Problems regarding : chargen (19/tcp)}
Security warnings :\\
\begin{itemize}
\item \begin{verbatim}
The chargen service is running.
The 'chargen' service should only be enabled when testing the machine.
 

When contacted, chargen responds with some random characters
 (something
like all the characters in the alphabet in a row). When contacted via
 UDP, it 
will respond with a single UDP packet. When contacted via TCP, it will
 
continue spewing characters until the client closes the connection. 

An easy attack is 'pingpong' in which an attacker spoofs a packet
 between two
machines running chargen. This will cause them to spew characters at
 each 
other, slowing the machines down and saturating the network.
      
Solution : disable this service in /etc/inetd.conf.

Risk factor : Low
CVE : CVE-1999-0103
\end{verbatim}\end{itemize}
Security note :\\
\begin{itemize}
\item \begin{verbatim}
Chargen is running on this port
\end{verbatim}\end{itemize}
\subsubsection{Problems regarding : qotd (17/tcp)}
Security warnings :\\
\begin{itemize}
\item \begin{verbatim}
The quote service (qotd) is running.

A server listens for TCP connections on TCP port 17. Once a connection
 
is established a short message is sent out the connection (and any 
data received is thrown away). The service closes the connection 
after sending the quote.

Another quote of the day service is defined as a datagram based
application on UDP.  A server listens for UDP datagrams on UDP port
 17.
When a datagram is received, an answering datagram is sent containing 
a quote (the data in the received datagram is ignored).


An easy attack is 'pingpong' which IP spoofs a packet between two
 machines
running qotd. This will cause them to spew characters at each other,
slowing the machines down and saturating the network.



Solution : disable this service in /etc/inetd.conf.

Risk factor : Low
CVE : CVE-1999-0103
\end{verbatim}\end{itemize}
Security note :\\
\begin{itemize}
\item \begin{verbatim}
qotd seems to be running on this port
\end{verbatim}\end{itemize}
\subsubsection{Problems regarding : daytime (13/tcp)}
Security warnings :\\
\begin{itemize}
\item \begin{verbatim}
The daytime service is running.
The date format issued by this service
may sometimes help an attacker to guess
the operating system type. 

In addition to that, when the UDP version of
daytime is running, an attacker may link it 
to the echo port using spoofing, thus creating
a possible denial of service.

Solution : disable this service in /etc/inetd.conf.

Risk factor : Low
CVE : CVE-1999-0103
\end{verbatim}\end{itemize}
\subsubsection{Problems regarding : echo (7/tcp)}
Security warnings :\\
\begin{itemize}
\item \begin{verbatim}
The 'echo' port is open. This port is
not of any use nowadays, and may be a source of problems, 
since it can be used along with other ports to perform a denial
of service. You should really disable this service.

Risk factor : Low

Solution : comment out 'echo' in /etc/inetd.conf
CVE : CVE-1999-0103
\end{verbatim}\end{itemize}
Security note :\\
\begin{itemize}
\item \begin{verbatim}
An echo server is running on this port
\end{verbatim}\end{itemize}
\subsubsection{Problems regarding : http (80/tcp)}
Security holes :\\
\begin{itemize}
\item \begin{verbatim}
The IIS server appears to have the .HTR ISAPI filter mapped.

At least one remote vulnerability has been discovered for the .HTR
filter. This is detailed in Microsoft Advisory
MS02-018, and gives remote SYSTEM level access to the web server. 

It is recommended that even if you have patched this vulnerability
 that
you unmap the .HTR extension, and any other unused ISAPI extensions
if they are not required for the operation of your site.

Solution: 
To unmap the .HTR extension:
 1.Open Internet Services Manager. 
 2.Right-click the Web server choose Properties from the context menu.
 
 3.Master Properties 
 4.Select WWW Service -> Edit -> HomeDirectory -> Configuration 
and remove the reference to .htr from the list.

Risk factor : High
CVE : CAN-2002-0071
BID : 4474
\end{verbatim}\item \begin{verbatim}
The web server is probably susceptible to a common IIS vulnerability
 discovered by
'Rain Forest Puppy'. This vulnerability enables an attacker to execute
 arbitrary
commands on the server with Administrator Privileges. 

*** Nessus solely relied on the presence of the file /msadc/msadcs.dll
*** so this might be a false positive

See Microsoft security bulletin (MS99-025) for patch information.
Also, BUGTRAQ ID 529 on www.securityfocus.com (
 http://www.securityfocus.com/bid/529 )

Risk factor : High
CVE : CVE-1999-1011
BID : 529
\end{verbatim}\end{itemize}
Security warnings :\\
\begin{itemize}
\item \begin{verbatim}
The IIS server appears to have the .IDA ISAPI filter mapped.

At least one remote vulnerability has been discovered for the .IDA
(indexing service) filter. This is detailed in Microsoft Advisory
MS01-033, and gives remote SYSTEM level access to the web server. 

It is recommended that even if you have patched this vulnerability
 that
you unmap the .IDA extension, and any other unused ISAPI extensions
if they are not required for the operation of your site.

Solution: 
To unmap the .IDA extension:
 1.Open Internet Services Manager. 
 2.Right-click the Web server choose Properties from the context menu.
 
 3.Master Properties 
 4.Select WWW Service -> Edit -> HomeDirectory -> Configuration 
and remove the reference to .ida from the list.

Risk factor : Medium
CVE : CAN-2002-0071
BID : 4474
\end{verbatim}\item \begin{verbatim}
IIS 5 has support for the Internet Printing Protocol(IPP), which is 
enabled in a default install. The protocol is implemented in IIS5 as
 an 
ISAPI extension. At least one security problem (a buffer overflow)
has been found with that extension in the past, so we recommend
you disable it if you do not use this functionality.

Solution: 
To unmap the .printer extension:
 1.Open Internet Services Manager. 
 2.Right-click the Web server choose Properties from the context menu.
 
 3.Master Properties 
 4.Select WWW Service -> Edit -> HomeDirectory -> Configuration 
and remove the reference to .printer from the list.

Reference : http://online.securityfocus.com/archive/1/181109

Risk factor : Low
\end{verbatim}\item \begin{verbatim}
Your webserver supports the TRACE and/or TRACK methods. It has been
shown that servers supporting this method are subject
to cross-site-scripting attacks, dubbed XST for
'Cross-Site-Tracing', when used in conjunction with
various weaknesses in browsers.

An attacker may use this flaw to trick your
legitimate web users to give him their 
credentials.

Solution: Disable these methods.


If you are using Apache, add the following lines for each virtual
host in your configuration file :

    RewriteEngine on
    RewriteCond %{REQUEST_METHOD} ^(TRACE|TRACK)
    RewriteRule .* - [F]

If you are using Microsoft IIS, use the URLScan tool to deny HTTP
 TRACE
requests or to permit only the methods needed to meet site
 requirements
and policy.



See http://www.whitehatsec.com/press_releases/WH-PR-20030120.pdf
    http://archives.neohapsis.com/archives/vulnwatch/2003-q1/0035.html

Risk factor : Medium
\end{verbatim}\item \begin{verbatim}
It is possible to retrieve the listing of the remote 
directories accessible via HTTP, rather than their index.html, 
using the Index Server service which provides WebDav capabilities
to this server.

This problem allows an attacker to gain more knowledge
about the remote host, and may make him aware of hidden
HTML files.

Solution : disable the Index Server service, or
see http://www.microsoft.com/technet/support/kb.asp?ID=272079
Risk factor : Low
CVE : CVE-2000-0951
BID : 1756
\end{verbatim}\item \begin{verbatim}
The remote web server appears to be running with
Frontpage extensions. 

You should double check the configuration since
a lot of security problems have been found with
FrontPage when the configuration file is
not well set up.

Risk factor : High if your configuration file is
not well set up
CVE : CAN-2000-0114
\end{verbatim}\end{itemize}
Security note :\\
\begin{itemize}
\item \begin{verbatim}
A web server is running on this port
\end{verbatim}\item \begin{verbatim}
The remote web server type is :

Microsoft-IIS/5.0

Solution : You can use urlscan to change reported server for IIS.
\end{verbatim}\item \begin{verbatim}
The following directories were discovered:
/_vti_bin, /images
The following directories require authentication:
/printers
\end{verbatim}\end{itemize}
\subsubsection{Problems regarding : nntp (119/tcp)}
Security note :\\
\begin{itemize}
\item \begin{verbatim}
An NNTP server is running on this port
\end{verbatim}\item \begin{verbatim}
Remote NNTP server version : 200 NNTP Service 5.00.0984 Version:
 5.0.2195.5329 Posting Allowed 

\end{verbatim}\item \begin{verbatim}
This NNTP server allows unauthenticated connections
For your information, we counted 4 newsgroups on this NNTP server:
0 in the alt hierarchy, 0 in rec, 0 in biz, 0 in sci, 0 in soc, 0 in
 misc, 0 in news, 0 in comp, 0 in talk, 0 in humanities.
Although this server says it allows posting, we were unable to send a
 message
(posted in alt.test)


\end{verbatim}\end{itemize}
\subsubsection{Problems regarding : loc-srv (135/tcp)}
Security warnings :\\
\begin{itemize}
\item \begin{verbatim}
DCE services running on the remote can be enumerated
by connecting on port 135 and doing the appropriate
queries.

An attacker may use this fact to gain more knowledge
about the remote host.

Solution : filter incoming traffic to this port.
Risk factor : Low
\end{verbatim}\end{itemize}
Security note :\\
\begin{itemize}
\item \begin{verbatim}
A DCE service is listening on this host
     UUID: 811109bf-a4e1-11d1-ab54-00a0c91e9b45, version 1
     Endpoint: ncacn_np:\\GABBO[\pipe\WinsPipe]


\end{verbatim}\item \begin{verbatim}
A DCE service is listening on this host
     UUID: 906b0ce0-c70b-1067-b317-00dd010662da, version 1
     Endpoint: ncalrpc[LRPC00000238.00000001]


\end{verbatim}\item \begin{verbatim}
A DCE service is listening on this host
     UUID: 906b0ce0-c70b-1067-b317-00dd010662da, version 1
     Endpoint: ncalrpc[LRPC00000238.00000001]


\end{verbatim}\item \begin{verbatim}
A DCE service is listening on this host
     UUID: 906b0ce0-c70b-1067-b317-00dd010662da, version 1
     Endpoint: ncalrpc[LRPC00000238.00000001]


\end{verbatim}\item \begin{verbatim}
A DCE service is listening on this host
     UUID: 906b0ce0-c70b-1067-b317-00dd010662da, version 1
     Endpoint: ncalrpc[LRPC00000238.00000001]


\end{verbatim}\item \begin{verbatim}
A DCE service is listening on this host
     UUID: 1ff70682-0a51-30e8-076d-740be8cee98b, version 1
     Endpoint: ncalrpc[LRPC000004a0.00000001]


\end{verbatim}\item \begin{verbatim}
A DCE service is listening on this host
     UUID: 378e52b0-c0a9-11cf-822d-00aa0051e40f, version 1
     Endpoint: ncalrpc[LRPC000004a0.00000001]


\end{verbatim}\item \begin{verbatim}
A DCE service is listening on this host
     UUID: 5a7b91f8-ff00-11d0-a9b2-00c04fb6e6fc, version 1
     Endpoint: ncalrpc[ntsvcs]
     Annotation: Messenger Service


\end{verbatim}\item \begin{verbatim}
A DCE service is listening on this host
     UUID: 5a7b91f8-ff00-11d0-a9b2-00c04fb6e6fc, version 1
     Endpoint: ncacn_np:\\GABBO[\PIPE\ntsvcs]
     Annotation: Messenger Service


\end{verbatim}\item \begin{verbatim}
A DCE service is listening on this host
     UUID: 5a7b91f8-ff00-11d0-a9b2-00c04fb6e6fc, version 1
     Endpoint: ncacn_np:\\GABBO[\PIPE\scerpc]
     Annotation: Messenger Service


\end{verbatim}\item \begin{verbatim}
A DCE service is listening on this host
     UUID: 5a7b91f8-ff00-11d0-a9b2-00c04fb6e6fc, version 1
     Endpoint: ncalrpc[DNSResolver]
     Annotation: Messenger Service


\end{verbatim}\item \begin{verbatim}
A DCE service is listening on this host
     UUID: 82ad4280-036b-11cf-972c-00aa006887b0, version 2
     Endpoint: ncalrpc[OLEc]


\end{verbatim}\item \begin{verbatim}
A DCE service is listening on this host
     UUID: 82ad4280-036b-11cf-972c-00aa006887b0, version 2
     Endpoint: ncalrpc[INETINFO_LPC]


\end{verbatim}\item \begin{verbatim}
A DCE service is listening on this host
     UUID: 82ad4280-036b-11cf-972c-00aa006887b0, version 2
     Endpoint: ncacn_np:\\GABBO[\PIPE\INETINFO]


\end{verbatim}\item \begin{verbatim}
A DCE service is listening on this host
     UUID: 8cfb5d70-31a4-11cf-a7d8-00805f48a135, version 3
     Endpoint: ncalrpc[OLEc]


\end{verbatim}\item \begin{verbatim}
A DCE service is listening on this host
     UUID: 8cfb5d70-31a4-11cf-a7d8-00805f48a135, version 3
     Endpoint: ncalrpc[INETINFO_LPC]


\end{verbatim}\item \begin{verbatim}
A DCE service is listening on this host
     UUID: 8cfb5d70-31a4-11cf-a7d8-00805f48a135, version 3
     Endpoint: ncacn_np:\\GABBO[\PIPE\INETINFO]


\end{verbatim}\item \begin{verbatim}
A DCE service is listening on this host
     UUID: 8cfb5d70-31a4-11cf-a7d8-00805f48a135, version 3
     Endpoint: ncalrpc[SMTPSVC_LPC]


\end{verbatim}\item \begin{verbatim}
A DCE service is listening on this host
     UUID: 8cfb5d70-31a4-11cf-a7d8-00805f48a135, version 3
     Endpoint: ncacn_np:\\GABBO[\PIPE\SMTPSVC]


\end{verbatim}\item \begin{verbatim}
A DCE service is listening on this host
     UUID: bfa951d1-2f0e-11d3-bfd1-00c04fa3490a, version 1
     Endpoint: ncalrpc[OLEc]


\end{verbatim}\item \begin{verbatim}
A DCE service is listening on this host
     UUID: bfa951d1-2f0e-11d3-bfd1-00c04fa3490a, version 1
     Endpoint: ncalrpc[INETINFO_LPC]


\end{verbatim}\item \begin{verbatim}
A DCE service is listening on this host
     UUID: bfa951d1-2f0e-11d3-bfd1-00c04fa3490a, version 1
     Endpoint: ncacn_np:\\GABBO[\PIPE\INETINFO]


\end{verbatim}\item \begin{verbatim}
A DCE service is listening on this host
     UUID: bfa951d1-2f0e-11d3-bfd1-00c04fa3490a, version 1
     Endpoint: ncalrpc[SMTPSVC_LPC]


\end{verbatim}\item \begin{verbatim}
A DCE service is listening on this host
     UUID: bfa951d1-2f0e-11d3-bfd1-00c04fa3490a, version 1
     Endpoint: ncacn_np:\\GABBO[\PIPE\SMTPSVC]


\end{verbatim}\item \begin{verbatim}
A DCE service is listening on this host
     UUID: bfa951d1-2f0e-11d3-bfd1-00c04fa3490a, version 1
     Endpoint: ncacn_at_dspGABBO[DynEpt 590.1]


\end{verbatim}\item \begin{verbatim}
A DCE service is listening on this host
     UUID: 4f82f460-0e21-11cf-909e-00805f48a135, version 4
     Endpoint: ncalrpc[OLEc]


\end{verbatim}\item \begin{verbatim}
A DCE service is listening on this host
     UUID: 4f82f460-0e21-11cf-909e-00805f48a135, version 4
     Endpoint: ncalrpc[INETINFO_LPC]


\end{verbatim}\item \begin{verbatim}
A DCE service is listening on this host
     UUID: 4f82f460-0e21-11cf-909e-00805f48a135, version 4
     Endpoint: ncacn_np:\\GABBO[\PIPE\INETINFO]


\end{verbatim}\item \begin{verbatim}
A DCE service is listening on this host
     UUID: 4f82f460-0e21-11cf-909e-00805f48a135, version 4
     Endpoint: ncalrpc[SMTPSVC_LPC]


\end{verbatim}\item \begin{verbatim}
A DCE service is listening on this host
     UUID: 4f82f460-0e21-11cf-909e-00805f48a135, version 4
     Endpoint: ncacn_np:\\GABBO[\PIPE\SMTPSVC]


\end{verbatim}\item \begin{verbatim}
A DCE service is listening on this host
     UUID: 4f82f460-0e21-11cf-909e-00805f48a135, version 4
     Endpoint: ncacn_at_dspGABBO[DynEpt 590.1]


\end{verbatim}\item \begin{verbatim}
A DCE service is listening on this host
     UUID: 4f82f460-0e21-11cf-909e-00805f48a135, version 4
     Endpoint: ncalrpc[NNTPSVC_LPC]


\end{verbatim}\item \begin{verbatim}
A DCE service is listening on this host
     UUID: 4f82f460-0e21-11cf-909e-00805f48a135, version 4
     Endpoint: ncacn_np:\\GABBO[\PIPE\NNTPSVC]


\end{verbatim}\item \begin{verbatim}
A DCE service is listening on this host
     UUID: 3d267954-eeb7-11d1-b94e-00c04fa3080d, version 1
     Endpoint: ncalrpc[LRPC00000504.00000001]


\end{verbatim}\item \begin{verbatim}
A DCE service is listening on this host
     UUID: 3d267954-eeb7-11d1-b94e-00c04fa3080d, version 1
     Endpoint: ncacn_np:\\GABBO[\pipe\HydraLsPipe]


\end{verbatim}\item \begin{verbatim}
A DCE service is listening on this host
     UUID: 12d4b7c8-77d5-11d1-8c24-00c04fa3080d, version 1
     Endpoint: ncalrpc[LRPC00000504.00000001]


\end{verbatim}\item \begin{verbatim}
A DCE service is listening on this host
     UUID: 12d4b7c8-77d5-11d1-8c24-00c04fa3080d, version 1
     Endpoint: ncacn_np:\\GABBO[\pipe\HydraLsPipe]


\end{verbatim}\item \begin{verbatim}
A DCE service is listening on this host
     UUID: 493c451c-155c-11d3-a314-00c04fb16103, version 1
     Endpoint: ncalrpc[LRPC00000504.00000001]


\end{verbatim}\item \begin{verbatim}
A DCE service is listening on this host
     UUID: 493c451c-155c-11d3-a314-00c04fb16103, version 1
     Endpoint: ncacn_np:\\GABBO[\pipe\HydraLsPipe]


\end{verbatim}\item \begin{verbatim}
A DCE service is listening on this host
     UUID: 45f52c28-7f9f-101a-b52b-08002b2efabe, version 1
     Endpoint: ncalrpc[LRPC0000053c.00000001]


\end{verbatim}\item \begin{verbatim}
A DCE service is listening on this host
     UUID: 45f52c28-7f9f-101a-b52b-08002b2efabe, version 1
     Endpoint: ncacn_np:\\GABBO[\pipe\WinsPipe]


\end{verbatim}\item \begin{verbatim}
A DCE service is listening on this host
     UUID: 811109bf-a4e1-11d1-ab54-00a0c91e9b45, version 1
     Endpoint: ncalrpc[LRPC0000053c.00000001]


\end{verbatim}\end{itemize}
\subsubsection{Problems regarding : netbios-ssn (139/tcp)}
Security holes :\\
\begin{itemize}
\item \begin{verbatim}
. It was possible to log into the remote host using a NULL session.
The concept of a NULL session is to provide a null username and
a null password, which grants the user the 'guest' access

To prevent null sessions, see MS KB Article Q143474 (NT 4.0) and
Q246261 (Windows 2000). 
Note that this won't completely disable null sessions, but will 
prevent them from connecting to IPC$
Please see
 http://msgs.securepoint.com/cgi-bin/get/nessus-0204/50/1.html

. All the smb tests will be done as ''/''
CVE : CVE-2000-0222
BID : 990
\end{verbatim}\end{itemize}
Security warnings :\\
\begin{itemize}
\item \begin{verbatim}
The domain SID can be obtained remotely. Its value is :

COOLDOMAIN : 0-0-0-0-0

An attacker can use it to obtain the list of the local users of this
 host
Solution : filter the ports 137 to 139 and 445
Risk factor : Low

CVE : CVE-2000-1200
BID : 959
\end{verbatim}\item \begin{verbatim}
The host SID can be obtained remotely. Its value is :

GABBO : 5-21-842925246-1563985344-2146861395

An attacker can use it to obtain the list of the local users of this
 host
Solution : filter the ports 137 to 139 and 445
Risk factor : Low

CVE : CVE-2000-1200
BID : 959
\end{verbatim}\item \begin{verbatim}
The host SID could be used to enumerate the names of the local users
of this host. 
(we only enumerated users name whose ID is between 1000 and 1020
for performance reasons)
This gives extra knowledge to an attacker, which
is not a good thing : 
- Administrator account name : Administrator (id 500)
- Guest account name : Guest (id 501)
- TsInternetUser (id 1000)
- NetShowServices (id 1001)
- NetShow Administrators (id 1002)
- IUSR_GABBO (id 1003)
- IWAM_GABBO (id 1004)
- DHCP Users (id 1005)
- DHCP Administrators (id 1006)
- WINS Users (id 1007)

Risk factor : Medium
Solution : filter incoming connections this port

CVE : CVE-2000-1200
BID : 959
\end{verbatim}\item \begin{verbatim}
Here is the browse list of the remote host : 

GABBO - 


This is potentially dangerous as this may help the attack
of a potential hacker by giving him extra targets to check for

Solution : filter incoming traffic to this port
Risk factor : Low

\end{verbatim}\end{itemize}
Security note :\\
\begin{itemize}
\item \begin{verbatim}
The remote native lan manager is : Windows 2000 LAN Manager
The remote Operating System is : Windows 5.0
The remote SMB Domain Name is : COOLDOMAIN


\end{verbatim}\end{itemize}
\subsubsection{Problems regarding : https (443/tcp)}
Security note :\\
\begin{itemize}
\item \begin{verbatim}
An unknown service is running on this port.
It is usually reserved for HTTPS
\end{verbatim}\end{itemize}
\subsubsection{Problems regarding : printer (515/tcp)}
Security note :\\
\begin{itemize}
\item \begin{verbatim}
An unknown server is running on this port.
If you know what it is, please send this banner to the Nessus team:
00: 01                                                 .              
                               


\end{verbatim}\end{itemize}
\subsubsection{Problems regarding : afpovertcp (548/tcp)}
Security note :\\
\begin{itemize}
\item \begin{verbatim}
This host is running an AppleShare File Services over IP.
  Machine type: Windows NT
  Server name: GABBO
  UAMs: ClearTxt Passwrd/Microsoft V1.0/MS2.0
  AFP Versions: AFPVersion 2.0/AFPVersion 2.1/AFP2.2

\end{verbatim}\end{itemize}
\subsubsection{Problems regarding : nntps (563/tcp)}
Security note :\\
\begin{itemize}
\item \begin{verbatim}
An unknown service is running on this port.
It is usually reserved for NNTPS
\end{verbatim}\end{itemize}
\subsubsection{Problems regarding : blackjack (1025/tcp)}
Security note :\\
\begin{itemize}
\item \begin{verbatim}
A DCE service is listening on this port
     UUID: 906b0ce0-c70b-1067-b317-00dd010662da, version 1
     Endpoint: ncacn_ip_tcp:10.163.156.9[1025]


\end{verbatim}\item \begin{verbatim}
A DCE service is listening on this port
     UUID: 906b0ce0-c70b-1067-b317-00dd010662da, version 1
     Endpoint: ncacn_ip_tcp:10.163.156.9[1025]


\end{verbatim}\item \begin{verbatim}
A DCE service is listening on this port
     UUID: 906b0ce0-c70b-1067-b317-00dd010662da, version 1
     Endpoint: ncacn_ip_tcp:10.163.156.9[1025]


\end{verbatim}\item \begin{verbatim}
A DCE service is listening on this port
     UUID: 906b0ce0-c70b-1067-b317-00dd010662da, version 1
     Endpoint: ncacn_ip_tcp:10.163.156.9[1025]


\end{verbatim}\end{itemize}
\subsubsection{Problems regarding : unknown (1028/tcp)}
Security note :\\
\begin{itemize}
\item \begin{verbatim}
A DCE service is listening on this port
     UUID: 1ff70682-0a51-30e8-076d-740be8cee98b, version 1
     Endpoint: ncacn_ip_tcp:10.163.156.9[1028]


\end{verbatim}\item \begin{verbatim}
A DCE service is listening on this port
     UUID: 378e52b0-c0a9-11cf-822d-00aa0051e40f, version 1
     Endpoint: ncacn_ip_tcp:10.163.156.9[1028]


\end{verbatim}\end{itemize}
\subsubsection{Problems regarding : unknown (1035/tcp)}
Security note :\\
\begin{itemize}
\item \begin{verbatim}
A DCE service is listening on this port
     UUID: 45f52c28-7f9f-101a-b52b-08002b2efabe, version 1
     Endpoint: ncacn_ip_tcp:10.163.156.9[1035]


\end{verbatim}\item \begin{verbatim}
A DCE service is listening on this port
     UUID: 811109bf-a4e1-11d1-ab54-00a0c91e9b45, version 1
     Endpoint: ncacn_ip_tcp:10.163.156.9[1035]


\end{verbatim}\end{itemize}
\subsubsection{Problems regarding : netinfo (1033/tcp)}
Security note :\\
\begin{itemize}
\item \begin{verbatim}
A DCE service is listening on this port
     UUID: 3d267954-eeb7-11d1-b94e-00c04fa3080d, version 1
     Endpoint: ncacn_ip_tcp:10.163.156.9[1033]


\end{verbatim}\item \begin{verbatim}
A DCE service is listening on this port
     UUID: 12d4b7c8-77d5-11d1-8c24-00c04fa3080d, version 1
     Endpoint: ncacn_ip_tcp:10.163.156.9[1033]


\end{verbatim}\item \begin{verbatim}
A DCE service is listening on this port
     UUID: 493c451c-155c-11d3-a314-00c04fb16103, version 1
     Endpoint: ncacn_ip_tcp:10.163.156.9[1033]


\end{verbatim}\end{itemize}
\subsubsection{Problems regarding : iad2 (1031/tcp)}
Security note :\\
\begin{itemize}
\item \begin{verbatim}
A DCE service is listening on this port
     UUID: 82ad4280-036b-11cf-972c-00aa006887b0, version 2
     Endpoint: ncacn_ip_tcp:10.163.156.9[1031]


\end{verbatim}\item \begin{verbatim}
A DCE service is listening on this port
     UUID: 8cfb5d70-31a4-11cf-a7d8-00805f48a135, version 3
     Endpoint: ncacn_ip_tcp:10.163.156.9[1031]


\end{verbatim}\item \begin{verbatim}
A DCE service is listening on this port
     UUID: bfa951d1-2f0e-11d3-bfd1-00c04fa3490a, version 1
     Endpoint: ncacn_ip_tcp:10.163.156.9[1031]


\end{verbatim}\item \begin{verbatim}
A DCE service is listening on this port
     UUID: 4f82f460-0e21-11cf-909e-00805f48a135, version 4
     Endpoint: ncacn_ip_tcp:10.163.156.9[1031]


\end{verbatim}\end{itemize}
\subsubsection{Problems regarding : ms-sql-s (1433/tcp)}
Security note :\\
\begin{itemize}
\item \begin{verbatim}
It is possible that Microsoft's SQL Server is installed on the remote
 computer.
CVE : CAN-1999-0652
\end{verbatim}\end{itemize}
\subsubsection{Problems regarding : ms-sql-m (1434/udp)}
Security warnings :\\
\begin{itemize}
\item \begin{verbatim}
Here is the reply to a MS SQL 'ping' request : ServerName;GABBO;InstanceName;MSSQLSERVER;IsClustered;No;Version;8.00.194;tcp;1433;np;\\GABBO\pipe\\sql\query;;
*** Note that the version number might be inaccurate, as Microsoft
*** decided to not increase it with new releases of its software
It is suggested you filter incoming traffic to this port


\end{verbatim}\end{itemize}
\subsubsection{Problems regarding : general/tcp}
Security warnings :\\
\begin{itemize}
\item \begin{verbatim}
The remote host uses non-random IP IDs, that is, it is
possible to predict the next value of the ip_id field of
the ip packets sent by this host.

An attacker may use this feature to determine if the remote
host sent a packet in reply to another request. This may be
used for portscanning and other things.

Solution : Contact your vendor for a patch
Risk factor : Low
\end{verbatim}\end{itemize}
Security note :\\
\begin{itemize}
\item \begin{verbatim}
QueSO has found out that the remote host OS is 
* WindowsNT, Cisco 11.2(10a), HP/3000 DTC, BayStack Switch


CVE : CAN-1999-0454
\end{verbatim}\end{itemize}
\subsubsection{Problems regarding : general/udp}
Security note :\\
\begin{itemize}
\item \begin{verbatim}
For your information, here is the traceroute to 10.163.156.9 : 
?
10.163.156.9

\end{verbatim}\end{itemize}
\subsubsection{Problems regarding : snmp (161/udp)}
Security holes :\\
\begin{itemize}
\item \begin{verbatim}
SNMP Agent responded as expected with community name: public
CVE : CAN-1999-0186
BID : 177
\end{verbatim}\end{itemize}
Security warnings :\\
\begin{itemize}
\item \begin{verbatim}
It was possible to obtain the list of SMB users of the
remote host via SNMP : 

. Guest
. IUSR_GABBO
. IWAM_GABBO
. Administrator
. TsInternetUser
. NetShowServices

An attacker may use this information to set up brute force
attacks or find an unused account.

Solution : disable the SNMP service on the remote host if you do not
           use it, or filter incoming UDP packets going to this port
Risk factor : Medium
\end{verbatim}\item \begin{verbatim}
It was possible to obtain the list of Lanman services of the
remote host via SNMP : 

. Server
. Alerter
. Event Log
. Messenger
. Telephony
. DNS Client
. DHCP Client
. MSSQLSERVER
. Workstation
. SNMP Service
. Plug and Play
. Print Spooler
. RunAs Service
. Task Scheduler
. Computer Browser
. Indexing Service
. Automatic Updates
. COM+ Event System
. IIS Admin Service
. Protected Storage
. Removable Storage
. Terminal Services
. IPSEC Policy Agent
. Remote Storage File
. TCP/IP Print Server
. Logical Disk Manager
. Remote Storage Media
. Remote Storage Engine
. FTP Publishing Service
. Simple TCP/IP Services
. Distributed File System
. License Logging Service
. Remote Registry Service
. File Server for Macintosh
. Security Accounts Manager
. System Event Notification
. Print Server for Macintosh
. Remote Procedure Call (RPC)
. Terminal Services Licensing
. TCP/IP NetBIOS Helper Service
. Windows Media Monitor Service
. Windows Media Program Service
. Windows Media Station Service
. Windows Media Unicast Service
. Internet Authentication Service
. NT LM Security Support Provider
. Distributed Link Tracking Client
. World Wide Web Publishing Service
. Windows Management Instrumentation
. Distributed Transaction Coordinator
. Windows Internet Name Service (WINS)
. Simple Mail Transport Protocol (SMTP)
. Network News Transport Protocol (NNTP)
. Windows Management Instrumentation Driver Extensions

An attacker may use this information to gain more knowledge about
the target host.
Solution : disable the SNMP service on the remote host if you do not
           use it, or filter incoming UDP packets going to this port
Risk factor : Low
\end{verbatim}\item \begin{verbatim}
It was possible to obtain the list of network interfaces of the
remote host via SNMP : 

. MS TCP Loopback interface
. Realtek RTL8029(AS) Ethernet Adapt

An attacker may use this information to gain more knowledge about
the target host.
Solution : disable the SNMP service on the remote host if you do not
           use it, or filter incoming UDP packets going to this port
Risk factor : Low
\end{verbatim}\end{itemize}
Security note :\\
\begin{itemize}
\item \begin{verbatim}
Using SNMP, we could determine that the remote operating system is :
Hardware: x86 Family 6 Model 6 Stepping 0 AT/AT COMPATIBLE - Software:
 Windows 2000 Version 5.0 (Build 2195 Uniprocessor Free)
\end{verbatim}\end{itemize}
\subsubsection{Problems regarding : netbios-ns (137/udp)}
Security warnings :\\
\begin{itemize}
\item \begin{verbatim}
. The following 9 NetBIOS names have been gathered :
 GABBO          
 GABBO          
 COOLDOMAIN     
 COOLDOMAIN     
 GABBO          
 COOLDOMAIN     
   __MSBROWSE__ 
 INet~Services  
 IS~GABBO       
. The remote host has the following MAC address on its adapter :
   0x00 0x40 0x05 0x65 0x01 0xa2 

If you do not want to allow everyone to find the NetBios name
of your computer, you should filter incoming traffic to this port.

Risk factor : Medium
\end{verbatim}\end{itemize}
\subsubsection{Problems regarding : echo (7/udp)}
Security warnings :\\
\begin{itemize}
\item \begin{verbatim}
The 'echo' port is open. This port is
not of any use nowadays, and may be a source of problems, 
since it can be used along with other ports to perform a denial
of service. You should really disable this service.

Risk factor : Low

Solution : comment out 'echo' in /etc/inetd.conf
CVE : CVE-1999-0103
\end{verbatim}\end{itemize}
\subsubsection{Problems regarding : ms-term-serv (3389/tcp)}
Security note :\\
\begin{itemize}
\item \begin{verbatim}
The Terminal Services are enabled on the remote host.

Terminal Services allow a Windows user to remotely obtain
a graphical login (and therefore act as a local user on the
remote host).

If an attacker gains a valid login and password, he may
be able to use this service to gain further access
on the remote host.


Solution : Disable the Terminal Services if you do not use them
Risk factor : Low
\end{verbatim}\end{itemize}
\subsubsection{Problems regarding : daytime (13/udp)}
Security warnings :\\
\begin{itemize}
\item \begin{verbatim}
The daytime service is running.
The date format issued by this service
may sometimes help an attacker to guess
the operating system type. 

In addition to that, when the UDP version of
daytime is running, an attacker may link it 
to the echo port using spoofing, thus creating
a possible denial of service.

Solution : disable this service in /etc/inetd.conf.

Risk factor : Low
CVE : CVE-1999-0103
\end{verbatim}\end{itemize}
\subsubsection{Problems regarding : qotd (17/udp)}
Security warnings :\\
\begin{itemize}
\item \begin{verbatim}
The quote service (qotd) is running.

A server listens for TCP connections on TCP port 17. Once a connection
 
is established a short message is sent out the connection (and any 
data received is thrown away). The service closes the connection 
after sending the quote.

Another quote of the day service is defined as a datagram based
application on UDP.  A server listens for UDP datagrams on UDP port
 17.
When a datagram is received, an answering datagram is sent containing 
a quote (the data in the received datagram is ignored).


An easy attack is 'pingpong' which IP spoofs a packet between two
 machines
running qotd. This will cause them to spew characters at each other,
slowing the machines down and saturating the network.



Solution : disable this service in /etc/inetd.conf.

Risk factor : Low
CVE : CVE-1999-0103
\end{verbatim}\end{itemize}
\subsubsection{Problems regarding : iad1 (1030/udp)}
Security note :\\
\begin{itemize}
\item \begin{verbatim}
A DCE service is listening on this port
     UUID: 5a7b91f8-ff00-11d0-a9b2-00c04fb6e6fc, version 1
     Endpoint: ncadg_ip_udp:10.163.156.9[1030]
     Annotation: Messenger Service


\end{verbatim}\end{itemize}
\subsubsection{Problems regarding : chargen (19/udp)}
Security warnings :\\
\begin{itemize}
\item \begin{verbatim}
The chargen service is running.
The 'chargen' service should only be enabled when testing the machine.
 

When contacted, chargen responds with some random characters
 (something
like all the characters in the alphabet in a row). When contacted via
 UDP, it 
will respond with a single UDP packet. When contacted via TCP, it will
 
continue spewing characters until the client closes the connection. 

An easy attack is 'pingpong' in which an attacker spoofs a packet
 between two
machines running chargen. This will cause them to spew characters at
 each 
other, slowing the machines down and saturating the network.
      
Solution : disable this service in /etc/inetd.conf.

Risk factor : Low
CVE : CVE-1999-0103
\end{verbatim}\end{itemize}
\subsubsection{Problems regarding : iad3 (1032/udp)}
Security note :\\
\begin{itemize}
\item \begin{verbatim}
A DCE service is listening on this port
     UUID: bfa951d1-2f0e-11d3-bfd1-00c04fa3490a, version 1
     Endpoint: ncadg_ip_udp:10.163.156.9[1032]


\end{verbatim}\item \begin{verbatim}
A DCE service is listening on this port
     UUID: 4f82f460-0e21-11cf-909e-00805f48a135, version 4
     Endpoint: ncadg_ip_udp:10.163.156.9[1032]


\end{verbatim}\end{itemize}
\newpage
\section{10.163.155.4}
\subsection{Open ports (TCP and UDP)}
\verb+10.163.155.4+ has the following ports that are open : 
\begin{itemize}
\item\verb+ftp (21/tcp)+
\item\verb+http (80/tcp)+
\item\verb+loc-srv (135/tcp)+
\item\verb+netbios-ssn (139/tcp)+
\item\verb+microsoft-ds (445/tcp)+
\item\verb+blackjack (1025/tcp)+
\item\verb+general/tcp+
\item\verb+general/udp+
\item\verb+netbios-ns (137/udp)+
\item\verb+unknown (1026/udp)+
\end{itemize}
You should disable the services that you do not use, as they are potential security flaws.
\subsection{Details of the vulnerabilities}
\subsubsection{Problems regarding : ftp (21/tcp)}
Security note :\\
\begin{itemize}
\item \begin{verbatim}
The service closed the connection after 0 seconds without sending any
 data
It might be protected by some TCP wrapper

\end{verbatim}\end{itemize}
\subsubsection{Problems regarding : http (80/tcp)}
Security holes :\\
\begin{itemize}
\item \begin{verbatim}
The remote proxy server seems to be ooops 1.4.6 or older.

This proxy is vulnerable to a buffer overflow that
allows an attacker to gain a shell on this host.

*** Note that this check made the remote proxy crash

Solution : Upgrade to the latest version of this software
Risk factor : High
CVE : CAN-2001-0029
BID : 2099
\end{verbatim}\end{itemize}
Security warnings :\\
\begin{itemize}
\item \begin{verbatim}
The misconfigured proxy accepts requests coming
from anywhere. This allows attackers to gain some anonymity when
 browsing 
some sensitive sites using your proxy, making the remote sites think
 that
the requests come from your network.

Solution: Reconfigure the remote proxy so that it only accepts
 requests coming 
from inside your network.
 
Risk factor : Low/Medium
\end{verbatim}\end{itemize}
Security note :\\
\begin{itemize}
\item \begin{verbatim}
A web server is running on this port
\end{verbatim}\item \begin{verbatim}
An HTTP proxy is running on this port
\end{verbatim}\end{itemize}
\subsubsection{Problems regarding : loc-srv (135/tcp)}
Security warnings :\\
\begin{itemize}
\item \begin{verbatim}
DCE services running on the remote can be enumerated
by connecting on port 135 and doing the appropriate
queries.

An attacker may use this fact to gain more knowledge
about the remote host.

Solution : filter incoming traffic to this port.
Risk factor : Low
\end{verbatim}\end{itemize}
Security note :\\
\begin{itemize}
\item \begin{verbatim}
A DCE service is listening on this host
     UUID: 1ff70682-0a51-30e8-076d-740be8cee98b, version 1
     Endpoint: ncalrpc[LRPC0000027c.00000001]


\end{verbatim}\item \begin{verbatim}
A DCE service is listening on this host
     UUID: 378e52b0-c0a9-11cf-822d-00aa0051e40f, version 1
     Endpoint: ncalrpc[LRPC0000027c.00000001]


\end{verbatim}\item \begin{verbatim}
A DCE service is listening on this host
     UUID: 5a7b91f8-ff00-11d0-a9b2-00c04fb6e6fc, version 1
     Endpoint: ncalrpc[ntsvcs]
     Annotation: Messenger Service


\end{verbatim}\item \begin{verbatim}
A DCE service is listening on this host
     UUID: 5a7b91f8-ff00-11d0-a9b2-00c04fb6e6fc, version 1
     Endpoint: ncacn_np:\\BENDER[\PIPE\ntsvcs]
     Annotation: Messenger Service


\end{verbatim}\item \begin{verbatim}
A DCE service is listening on this host
     UUID: 5a7b91f8-ff00-11d0-a9b2-00c04fb6e6fc, version 1
     Endpoint: ncacn_np:\\BENDER[\PIPE\scerpc]
     Annotation: Messenger Service


\end{verbatim}\end{itemize}
\subsubsection{Problems regarding : netbios-ssn (139/tcp)}
Security holes :\\
\begin{itemize}
\item \begin{verbatim}
. It was possible to log into the remote host using a NULL session.
The concept of a NULL session is to provide a null username and
a null password, which grants the user the 'guest' access

To prevent null sessions, see MS KB Article Q143474 (NT 4.0) and
Q246261 (Windows 2000). 
Note that this won't completely disable null sessions, but will 
prevent them from connecting to IPC$
Please see
 http://msgs.securepoint.com/cgi-bin/get/nessus-0204/50/1.html

. All the smb tests will be done as ''/'' in domain WORKGROUP      
CVE : CVE-2000-0222
BID : 990
\end{verbatim}\end{itemize}
Security warnings :\\
\begin{itemize}
\item \begin{verbatim}
The domain SID can be obtained remotely. Its value is :

WORKGROUP : 0-0-0-0-0

An attacker can use it to obtain the list of the local users of this
 host
Solution : filter the ports 137 to 139 and 445
Risk factor : Low

CVE : CVE-2000-1200
BID : 959
\end{verbatim}\item \begin{verbatim}
The host SID can be obtained remotely. Its value is :

BENDER : 5-21-1884898659-186063924-2090620667

An attacker can use it to obtain the list of the local users of this
 host
Solution : filter the ports 137 to 139 and 445
Risk factor : Low

CVE : CVE-2000-1200
BID : 959
\end{verbatim}\item \begin{verbatim}
The host SID could be used to enumerate the names of the local users
of this host. 
(we only enumerated users name whose ID is between 1000 and 1020
for performance reasons)
This gives extra knowledge to an attacker, which
is not a good thing : 
- Administrator account name : Administrateur (id 500)
- Guest account name : Invit  (id 501)
- Renaud (id 1000)

Risk factor : Medium
Solution : filter incoming connections this port

CVE : CVE-2000-1200
BID : 959
\end{verbatim}\item \begin{verbatim}
Here is the browse list of the remote host : 

BENDER - 
XP - 


This is potentially dangerous as this may help the attack
of a potential hacker by giving him extra targets to check for

Solution : filter incoming traffic to this port
Risk factor : Low

\end{verbatim}\end{itemize}
Security note :\\
\begin{itemize}
\item \begin{verbatim}
The remote native lan manager is : Windows 2000 LAN Manager
The remote Operating System is : Windows 5.0
The remote SMB Domain Name is : WORKGROUP


\end{verbatim}\end{itemize}
\subsubsection{Problems regarding : blackjack (1025/tcp)}
Security note :\\
\begin{itemize}
\item \begin{verbatim}
A DCE service is listening on this port
     UUID: 1ff70682-0a51-30e8-076d-740be8cee98b, version 1
     Endpoint: ncacn_ip_tcp:10.163.155.4[1025]


\end{verbatim}\item \begin{verbatim}
A DCE service is listening on this port
     UUID: 378e52b0-c0a9-11cf-822d-00aa0051e40f, version 1
     Endpoint: ncacn_ip_tcp:10.163.155.4[1025]


\end{verbatim}\end{itemize}
\subsubsection{Problems regarding : general/tcp}
Security note :\\
\begin{itemize}
\item \begin{verbatim}
QueSO has found out that the remote host OS is 
* FreeBSD, NetBSD, OpenBSD  


CVE : CAN-1999-0454
\end{verbatim}\end{itemize}
\subsubsection{Problems regarding : general/udp}
Security note :\\
\begin{itemize}
\item \begin{verbatim}
For your information, here is the traceroute to 10.163.155.4 : 
?
10.163.156.1
10.163.155.4

\end{verbatim}\end{itemize}
\subsubsection{Problems regarding : netbios-ns (137/udp)}
Security warnings :\\
\begin{itemize}
\item \begin{verbatim}
. The following 5 NetBIOS names have been gathered :
 BENDER          = This is the computer name registered for
 workstation services by a WINS client.
 WORKGROUP       = Workgroup / Domain name
 BENDER         
 BENDER          = Computer name that is registered for the messenger
 service on a computer that is a WINS client.
 WORKGROUP       = Workgroup / Domain name (part of the Browser
 elections)
. The remote host has the following MAC address on its adapter :
   0x00 0x02 0x2d 0x28 0xf3 0x16 

If you do not want to allow everyone to find the NetBios name
of your computer, you should filter incoming traffic to this port.

Risk factor : Medium
\end{verbatim}\end{itemize}
\subsubsection{Problems regarding : unknown (1026/udp)}
Security note :\\
\begin{itemize}
\item \begin{verbatim}
A DCE service is listening on this port
     UUID: 5a7b91f8-ff00-11d0-a9b2-00c04fb6e6fc, version 1
     Endpoint: ncadg_ip_udp:10.163.155.4[1026]
     Annotation: Messenger Service


\end{verbatim}\end{itemize}
\newpage
\section{10.163.155.3}
\subsection{Open ports (TCP and UDP)}
\verb+10.163.155.3+ has the following ports that are open : 
\begin{itemize}
\item\verb+ftp (21/tcp)+
\item\verb+http (80/tcp)+
\item\verb+svrloc (427/tcp)+
\item\verb+afpovertcp (548/tcp)+
\item\verb+general/tcp+
\item\verb+general/udp+
\item\verb+x11 (6000/tcp)+
\end{itemize}
You should disable the services that you do not use, as they are potential security flaws.
\subsection{Details of the vulnerabilities}
\subsubsection{Problems regarding : ftp (21/tcp)}
Security holes :\\
\begin{itemize}
\item \begin{verbatim}
It was possible to make the remote FTP server
crash by issuing this command :

 CEL aaaa[...]aaaa

This problem is known has the 'AIX FTPd' overflow and
may allow the remote user to easily gain access to the 
root (super-user) account on the remote system.

Solution : If you are using AIX FTPd, then read
IBM's advisory number ERS-SVA-E01-1999:004.1,
or contact your vendor for a patch.

Risk factor : High
CVE : CVE-1999-0789
BID : 679
\end{verbatim}\item \begin{verbatim}
The remote FTP server closes
the connection when one of the commands is given
a too long argument. 

This probably due to a buffer overflow, which
allows anyone to execute arbitrary code
on the remote host.

This problem is threatening, because
the attackers don't need an account 
to exploit this flaw.

Solution : Upgrade your FTP server or change it
Risk factor : High
CVE : CAN-2000-0133
BID : 961
\end{verbatim}\end{itemize}
Security note :\\
\begin{itemize}
\item \begin{verbatim}
An FTP server is running on this port.
Here is its banner : 
220 10.163.155.3 FTP server (lukemftpd 1.1) ready.
\end{verbatim}\item \begin{verbatim}
Remote FTP server banner :
220 10.163.155.3 FTP server (lukemftpd 1.1) ready.
\end{verbatim}\end{itemize}
\subsubsection{Problems regarding : http (80/tcp)}
Security warnings :\\
\begin{itemize}
\item \begin{verbatim}
Your webserver supports the TRACE and/or TRACK methods. It has been
shown that servers supporting this method are subject
to cross-site-scripting attacks, dubbed XST for
'Cross-Site-Tracing', when used in conjunction with
various weaknesses in browsers.

An attacker may use this flaw to trick your
legitimate web users to give him their 
credentials.

Solution: Disable these methods.


If you are using Apache, add the following lines for each virtual
host in your configuration file :

    RewriteEngine on
    RewriteCond %{REQUEST_METHOD} ^(TRACE|TRACK)
    RewriteRule .* - [F]

If you are using Microsoft IIS, use the URLScan tool to deny HTTP
 TRACE
requests or to permit only the methods needed to meet site
 requirements
and policy.



See http://www.whitehatsec.com/press_releases/WH-PR-20030120.pdf
    http://archives.neohapsis.com/archives/vulnwatch/2003-q1/0035.html

Risk factor : Medium
\end{verbatim}\item \begin{verbatim}
The misconfigured proxy accepts requests coming
from anywhere. This allows attackers to gain some anonymity when
 browsing 
some sensitive sites using your proxy, making the remote sites think
 that
the requests come from your network.

Solution: Reconfigure the remote proxy so that it only accepts
 requests coming 
from inside your network.
 
Risk factor : Low/Medium
\end{verbatim}\end{itemize}
Security note :\\
\begin{itemize}
\item \begin{verbatim}
A web server is running on this port
\end{verbatim}\item \begin{verbatim}
An HTTP proxy is running on this port
\end{verbatim}\item \begin{verbatim}
The remote web server type is :

squid/2.5.PRE13

Solution : We recommend that you configure (if possible) your web
 server to return
a bogus Server header in order to not leak information.

\end{verbatim}\item \begin{verbatim}
This port was detected as being open by a port scanner but is now
 closed.
This service might have been crashed by a port scanner or by some
 information gathering plugin


\end{verbatim}\end{itemize}
\subsubsection{Problems regarding : svrloc (427/tcp)}
Security note :\\
\begin{itemize}
\item \begin{verbatim}
An unknown server is running on this port.
If you know what it is, please send this banner to the Nessus team:
00: 02 02                                              ..             
                             


\end{verbatim}\end{itemize}
\subsubsection{Problems regarding : afpovertcp (548/tcp)}
Security holes :\\
\begin{itemize}
\item \begin{verbatim}
This AppleShare File Server allows the 'guest' user to connect.

\end{verbatim}\end{itemize}
Security note :\\
\begin{itemize}
\item \begin{verbatim}
This host is running an AppleShare File Services over IP.
  Machine type: Macintosh
  Server name: betrayal
  UAMs: DHCAST128/DHX2/Cleartxt Passwrd/No User Authent
  AFP Versions: AFP3.1/AFPX03/AFP2.2/AFPVersion 2.1/AFPVersion
 2.0/AFPVersion 1.1

\end{verbatim}\end{itemize}
\subsubsection{Problems regarding : general/tcp}
Security note :\\
\begin{itemize}
\item \begin{verbatim}
QueSO has found out that the remote host OS is 
* FreeBSD, NetBSD, OpenBSD  


CVE : CAN-1999-0454
\end{verbatim}\end{itemize}
\subsubsection{Problems regarding : general/udp}
Security note :\\
\begin{itemize}
\item \begin{verbatim}
For your information, here is the traceroute to 10.163.155.3 : 
10.163.155.3

\end{verbatim}\end{itemize}
\subsubsection{Problems regarding : x11 (6000/tcp)}
Security warnings :\\
\begin{itemize}
\item \begin{verbatim}
This X server does *not* allow any client to connect to it
however it is recommended that you filter incoming connections
to this port as attacker may send garbage data and slow down
your X session or even kill the server.

Here is the server version : 11.0
Here is the message we received : No protocol specified


Solution : filter incoming connections to ports 6000-6009
Risk factor : Low
CVE : CVE-1999-0526
\end{verbatim}\end{itemize}
\newpage
\section{10.163.155.2}
\subsection{Open ports (TCP and UDP)}
\verb+10.163.155.2+ has the following ports that are open : 
\begin{itemize}
\item\verb+ftp (21/tcp)+
\item\verb+http (80/tcp)+
\item\verb+snmp (161/udp)+
\item\verb+general/tcp+
\end{itemize}
You should disable the services that you do not use, as they are potential security flaws.
\subsection{Details of the vulnerabilities}
\subsubsection{Problems regarding : ftp (21/tcp)}
Security note :\\
\begin{itemize}
\item \begin{verbatim}
The service closed the connection after 0 seconds without sending any
 data
It might be protected by some TCP wrapper

\end{verbatim}\end{itemize}
\subsubsection{Problems regarding : http (80/tcp)}
Security warnings :\\
\begin{itemize}
\item \begin{verbatim}
Your webserver supports the TRACE and/or TRACK methods. It has been
shown that servers supporting this method are subject
to cross-site-scripting attacks, dubbed XST for
'Cross-Site-Tracing', when used in conjunction with
various weaknesses in browsers.

An attacker may use this flaw to trick your
legitimate web users to give him their 
credentials.

Solution: Disable these methods.


If you are using Apache, add the following lines for each virtual
host in your configuration file :

    RewriteEngine on
    RewriteCond %{REQUEST_METHOD} ^(TRACE|TRACK)
    RewriteRule .* - [F]

If you are using Microsoft IIS, use the URLScan tool to deny HTTP
 TRACE
requests or to permit only the methods needed to meet site
 requirements
and policy.



See http://www.whitehatsec.com/press_releases/WH-PR-20030120.pdf
    http://archives.neohapsis.com/archives/vulnwatch/2003-q1/0035.html

Risk factor : Medium
\end{verbatim}\item \begin{verbatim}
The misconfigured proxy accepts requests coming
from anywhere. This allows attackers to gain some anonymity when
 browsing 
some sensitive sites using your proxy, making the remote sites think
 that
the requests come from your network.

Solution: Reconfigure the remote proxy so that it only accepts
 requests coming 
from inside your network.
 
Risk factor : Low/Medium
\end{verbatim}\end{itemize}
Security note :\\
\begin{itemize}
\item \begin{verbatim}
A web server is running on this port
\end{verbatim}\item \begin{verbatim}
An HTTP proxy is running on this port
\end{verbatim}\item \begin{verbatim}
The remote web server type is :

squid/2.5.PRE13

Solution : We recommend that you configure (if possible) your web
 server to return
a bogus Server header in order to not leak information.

\end{verbatim}\end{itemize}
\subsubsection{Problems regarding : snmp (161/udp)}
Security holes :\\
\begin{itemize}
\item \begin{verbatim}
The device answered to more than 4 community strings.
This may be a false positive or a community-less SNMP server
HP printers answer to all community strings.

SNMP Agent responded as expected with community name: public
SNMP Agent responded as expected with community name: private
SNMP Agent responded as expected with community name: ilmi
SNMP Agent responded as expected with community name: ILMI If the
 target is a Cisco Product, please read
 http://www.cisco.com/warp/public/707/ios-snmp-ilmi-vuln-pub.shtml
SNMP Agent responded as expected with community name: system
SNMP Agent responded as expected with community name: write
SNMP Agent responded as expected with community name: all
SNMP Agent responded as expected with community name: monitor
SNMP Agent responded as expected with community name: agent
SNMP Agent responded as expected with community name: manager
SNMP Agent responded as expected with community name: OrigEquipMfr
SNMP Agent responded as expected with community name: admin
SNMP Agent responded as expected with community name: default
SNMP Agent responded as expected with community name: password
SNMP Agent responded as expected with community name: tivoli
SNMP Agent responded as expected with community name: openview
SNMP Agent responded as expected with community name: community
SNMP Agent responded as expected with community name: snmp
SNMP Agent responded as expected with community name: snmpd
SNMP Agent responded as expected with community name: Secret C0de
SNMP Agent responded as expected with community name: security
SNMP Agent responded as expected with community name: all private
SNMP Agent responded as expected with community name: rmon
SNMP Agent responded as expected with community name: rmon_admin
SNMP Agent responded as expected with community name: hp_admin
SNMP Agent responded as expected with community name: NoGaH$@!
SNMP Agent responded as expected with community name: 0392a0
SNMP Agent responded as expected with community name: xyzzy
SNMP Agent responded as expected with community name: agent_steal
SNMP Agent responded as expected with community name: freekevin
SNMP Agent responded as expected with community name: fubar
SNMP Agent responded as expected with community name: secret
SNMP Agent responded as expected with community name: cisco
SNMP Agent responded as expected with community name: apc
SNMP Agent responded as expected with community name: ANYCOM
SNMP Agent responded as expected with community name: cable-docsis
SNMP Agent responded as expected with community name: c
SNMP Agent responded as expected with community name: cc
SNMP Agent responded as expected with community name: Cisco router
SNMP Agent responded as expected with community name: cascade
SNMP Agent responded as expected with community name: comcomcom
CVE : CAN-1999-0186
BID : 177
\end{verbatim}\end{itemize}
Security warnings :\\
\begin{itemize}
\item \begin{verbatim}
A SNMP server is running on this host
\end{verbatim}\end{itemize}
Security note :\\
\begin{itemize}
\item \begin{verbatim}
Using SNMP, we could determine that the remote operating system is :
Base Station V3.81 Compatible
\end{verbatim}\end{itemize}
\subsubsection{Problems regarding : general/tcp}
Security warnings :\\
\begin{itemize}
\item \begin{verbatim}
The remote host is a Wireless Access Point.
You should ensure that the proper physical and logical controls exist
around the AP.

Risk factor : Medium/Low
\end{verbatim}\end{itemize}
\newpage
\section{10.163.156.1}
\subsection{Open ports (TCP and UDP)}
\verb+10.163.156.1+ has the following ports that are open : 
\begin{itemize}
\item\verb+ssh (22/tcp)+
\item\verb+ftp (21/tcp)+
\item\verb+http (80/tcp)+
\item\verb+general/tcp+
\item\verb+general/udp+
\end{itemize}
You should disable the services that you do not use, as they are potential security flaws.
\subsection{Details of the vulnerabilities}
\subsubsection{Problems regarding : ssh (22/tcp)}
Security warnings :\\
\begin{itemize}
\item \begin{verbatim}
The remote SSH daemon supports connections made
using the version 1.33 and/or 1.5 of the SSH protocol.

These protocols are not completely cryptographically
safe so they should not be used.

Solution : 
 If you use OpenSSH, set the option 'Protocol' to '2'
 If you use SSH.com's set the option 'Ssh1Compatibility' to 'no'
  
Risk factor : Low
\end{verbatim}\end{itemize}
Security note :\\
\begin{itemize}
\item \begin{verbatim}
An ssh server is running on this port
\end{verbatim}\item \begin{verbatim}
The remote SSH daemon supports the following versions of the
SSH protocol :

  . 1.33
  . 1.5
  . 1.99
  . 2.0

\end{verbatim}\item \begin{verbatim}
Remote SSH version : SSH-1.99-OpenSSH_3.5
\end{verbatim}\end{itemize}
\subsubsection{Problems regarding : ftp (21/tcp)}
Security note :\\
\begin{itemize}
\item \begin{verbatim}
The service closed the connection after 0 seconds without sending any
 data
It might be protected by some TCP wrapper

\end{verbatim}\end{itemize}
\subsubsection{Problems regarding : http (80/tcp)}
Security holes :\\
\begin{itemize}
\item \begin{verbatim}
The remote proxy server seems to be ooops 1.4.6 or older.

This proxy is vulnerable to a buffer overflow that
allows an attacker to gain a shell on this host.

*** Note that this check made the remote proxy crash

Solution : Upgrade to the latest version of this software
Risk factor : High
CVE : CAN-2001-0029
BID : 2099
\end{verbatim}\end{itemize}
Security note :\\
\begin{itemize}
\item \begin{verbatim}
A web server is running on this port
\end{verbatim}\item \begin{verbatim}
An HTTP proxy is running on this port
\end{verbatim}\end{itemize}
\subsubsection{Problems regarding : general/tcp}
Security note :\\
\begin{itemize}
\item \begin{verbatim}
QueSO has found out that the remote host OS is 
* FreeBSD, NetBSD, OpenBSD  


CVE : CAN-1999-0454
\end{verbatim}\end{itemize}
\subsubsection{Problems regarding : general/udp}
Security note :\\
\begin{itemize}
\item \begin{verbatim}
For your information, here is the traceroute to 10.163.156.1 : 
?
10.163.156.1

\end{verbatim}\end{itemize}
\newpage
\section{10.163.155.6}
\subsection{Open ports (TCP and UDP)}
\verb+10.163.155.6+ has the following ports that are open : 
\begin{itemize}
\item\verb+ftp (21/tcp)+
\item\verb+http (80/tcp)+
\item\verb+loc-srv (135/tcp)+
\item\verb+netbios-ssn (139/tcp)+
\item\verb+microsoft-ds (445/tcp)+
\item\verb+blackjack (1025/tcp)+
\item\verb+general/tcp+
\item\verb+general/udp+
\item\verb+netbios-ns (137/udp)+
\item\verb+ms-term-serv (3389/tcp)+
\item\verb+unknown (1027/udp)+
\end{itemize}
You should disable the services that you do not use, as they are potential security flaws.
\subsection{Details of the vulnerabilities}
\subsubsection{Problems regarding : ftp (21/tcp)}
Security note :\\
\begin{itemize}
\item \begin{verbatim}
The service closed the connection after 0 seconds without sending any
 data
It might be protected by some TCP wrapper

\end{verbatim}\end{itemize}
\subsubsection{Problems regarding : http (80/tcp)}
Security holes :\\
\begin{itemize}
\item \begin{verbatim}
It was possible to kill the web server by
sending an invalid request with a too long HTTP 1.1 header
(Accept-Encoding, Accept-Language, Accept-Range, Connection, 
Expect, If-Match, If-None-Match, If-Range, If-Unmodified-Since,
Max-Forwards, TE, Host)

A cracker may exploit this vulnerability to make your web server
crash continually or even execute arbirtray code on your system.

Solution : upgrade your software or protect it with a filtering
 reverse proxy
Risk factor : High
\end{verbatim}\end{itemize}
Security warnings :\\
\begin{itemize}
\item \begin{verbatim}
Your webserver supports the TRACE and/or TRACK methods. It has been
shown that servers supporting this method are subject
to cross-site-scripting attacks, dubbed XST for
'Cross-Site-Tracing', when used in conjunction with
various weaknesses in browsers.

An attacker may use this flaw to trick your
legitimate web users to give him their 
credentials.

Solution: Disable these methods.


If you are using Apache, add the following lines for each virtual
host in your configuration file :

    RewriteEngine on
    RewriteCond %{REQUEST_METHOD} ^(TRACE|TRACK)
    RewriteRule .* - [F]

If you are using Microsoft IIS, use the URLScan tool to deny HTTP
 TRACE
requests or to permit only the methods needed to meet site
 requirements
and policy.



See http://www.whitehatsec.com/press_releases/WH-PR-20030120.pdf
    http://archives.neohapsis.com/archives/vulnwatch/2003-q1/0035.html

Risk factor : Medium
\end{verbatim}\item \begin{verbatim}
The misconfigured proxy accepts requests coming
from anywhere. This allows attackers to gain some anonymity when
 browsing 
some sensitive sites using your proxy, making the remote sites think
 that
the requests come from your network.

Solution: Reconfigure the remote proxy so that it only accepts
 requests coming 
from inside your network.
 
Risk factor : Low/Medium
\end{verbatim}\end{itemize}
Security note :\\
\begin{itemize}
\item \begin{verbatim}
A web server is running on this port
\end{verbatim}\item \begin{verbatim}
An HTTP proxy is running on this port
\end{verbatim}\item \begin{verbatim}
The remote web server type is :

squid/2.5.PRE13

Solution : We recommend that you configure (if possible) your web
 server to return
a bogus Server header in order to not leak information.

\end{verbatim}\item \begin{verbatim}
This port was detected as being open by a port scanner but is now
 closed.
This service might have been crashed by a port scanner or by some
 information gathering plugin


\end{verbatim}\end{itemize}
\subsubsection{Problems regarding : loc-srv (135/tcp)}
Security warnings :\\
\begin{itemize}
\item \begin{verbatim}
DCE services running on the remote can be enumerated
by connecting on port 135 and doing the appropriate
queries.

An attacker may use this fact to gain more knowledge
about the remote host.

Solution : filter incoming traffic to this port.
Risk factor : Low
\end{verbatim}\end{itemize}
Security note :\\
\begin{itemize}
\item \begin{verbatim}
A DCE service is listening on this host
     UUID: 1ff70682-0a51-30e8-076d-740be8cee98b, version 1
     Endpoint: ncalrpc[wzcsvc]


\end{verbatim}\item \begin{verbatim}
A DCE service is listening on this host
     UUID: 1ff70682-0a51-30e8-076d-740be8cee98b, version 1
     Endpoint: ncalrpc[OLE3]


\end{verbatim}\item \begin{verbatim}
A DCE service is listening on this host
     UUID: 1ff70682-0a51-30e8-076d-740be8cee98b, version 1
     Endpoint: ncacn_np:\\XP[\PIPE\atsvc]


\end{verbatim}\item \begin{verbatim}
A DCE service is listening on this host
     UUID: 378e52b0-c0a9-11cf-822d-00aa0051e40f, version 1
     Endpoint: ncalrpc[wzcsvc]


\end{verbatim}\item \begin{verbatim}
A DCE service is listening on this host
     UUID: 378e52b0-c0a9-11cf-822d-00aa0051e40f, version 1
     Endpoint: ncalrpc[OLE3]


\end{verbatim}\item \begin{verbatim}
A DCE service is listening on this host
     UUID: 378e52b0-c0a9-11cf-822d-00aa0051e40f, version 1
     Endpoint: ncacn_np:\\XP[\PIPE\atsvc]


\end{verbatim}\item \begin{verbatim}
A DCE service is listening on this host
     UUID: 0a74ef1c-41a4-4e06-83ae-dc74fb1cdd53, version 1
     Endpoint: ncalrpc[wzcsvc]


\end{verbatim}\item \begin{verbatim}
A DCE service is listening on this host
     UUID: 0a74ef1c-41a4-4e06-83ae-dc74fb1cdd53, version 1
     Endpoint: ncalrpc[OLE3]


\end{verbatim}\item \begin{verbatim}
A DCE service is listening on this host
     UUID: 0a74ef1c-41a4-4e06-83ae-dc74fb1cdd53, version 1
     Endpoint: ncacn_np:\\XP[\PIPE\atsvc]


\end{verbatim}\item \begin{verbatim}
A DCE service is listening on this host
     UUID: 5a7b91f8-ff00-11d0-a9b2-00c04fb6e6fc, version 1
     Endpoint: ncalrpc[wzcsvc]
     Annotation: Messenger Service


\end{verbatim}\item \begin{verbatim}
A DCE service is listening on this host
     UUID: 5a7b91f8-ff00-11d0-a9b2-00c04fb6e6fc, version 1
     Endpoint: ncalrpc[OLE3]
     Annotation: Messenger Service


\end{verbatim}\item \begin{verbatim}
A DCE service is listening on this host
     UUID: 5a7b91f8-ff00-11d0-a9b2-00c04fb6e6fc, version 1
     Endpoint: ncacn_np:\\XP[\PIPE\atsvc]
     Annotation: Messenger Service


\end{verbatim}\item \begin{verbatim}
A DCE service is listening on this host
     UUID: 5a7b91f8-ff00-11d0-a9b2-00c04fb6e6fc, version 1
     Endpoint: ncacn_np:\\XP[\PIPE\AudioSrv]
     Annotation: Messenger Service


\end{verbatim}\item \begin{verbatim}
A DCE service is listening on this host
     UUID: 5a7b91f8-ff00-11d0-a9b2-00c04fb6e6fc, version 1
     Endpoint: ncacn_np:\\XP[\PIPE\wkssvc]
     Annotation: Messenger Service


\end{verbatim}\item \begin{verbatim}
A DCE service is listening on this host
     UUID: 5a7b91f8-ff00-11d0-a9b2-00c04fb6e6fc, version 1
     Endpoint: ncacn_np:\\XP[\PIPE\SECLOGON]
     Annotation: Messenger Service


\end{verbatim}\item \begin{verbatim}
A DCE service is listening on this host
     UUID: 5a7b91f8-ff00-11d0-a9b2-00c04fb6e6fc, version 1
     Endpoint: ncacn_np:\\XP[\pipe\trkwks]
     Annotation: Messenger Service


\end{verbatim}\item \begin{verbatim}
A DCE service is listening on this host
     UUID: 5a7b91f8-ff00-11d0-a9b2-00c04fb6e6fc, version 1
     Endpoint: ncalrpc[trkwks]
     Annotation: Messenger Service


\end{verbatim}\item \begin{verbatim}
A DCE service is listening on this host
     UUID: 5a7b91f8-ff00-11d0-a9b2-00c04fb6e6fc, version 1
     Endpoint: ncacn_np:\\XP[\PIPE\W32TIME]
     Annotation: Messenger Service


\end{verbatim}\item \begin{verbatim}
A DCE service is listening on this host
     UUID: 5a7b91f8-ff00-11d0-a9b2-00c04fb6e6fc, version 1
     Endpoint: ncacn_np:\\XP[\pipe\keysvc]
     Annotation: Messenger Service


\end{verbatim}\item \begin{verbatim}
A DCE service is listening on this host
     UUID: 5a7b91f8-ff00-11d0-a9b2-00c04fb6e6fc, version 1
     Endpoint: ncalrpc[keysvc]
     Annotation: Messenger Service


\end{verbatim}\item \begin{verbatim}
A DCE service is listening on this host
     UUID: 5a7b91f8-ff00-11d0-a9b2-00c04fb6e6fc, version 1
     Endpoint: ncalrpc[senssvc]
     Annotation: Messenger Service


\end{verbatim}\item \begin{verbatim}
A DCE service is listening on this host
     UUID: 5a7b91f8-ff00-11d0-a9b2-00c04fb6e6fc, version 1
     Endpoint: ncacn_np:\\XP[\PIPE\srvsvc]
     Annotation: Messenger Service


\end{verbatim}\item \begin{verbatim}
A DCE service is listening on this host
     UUID: 5a7b91f8-ff00-11d0-a9b2-00c04fb6e6fc, version 1
     Endpoint: ncalrpc[srrpc]
     Annotation: Messenger Service


\end{verbatim}\item \begin{verbatim}
A DCE service is listening on this host
     UUID: 5a7b91f8-ff00-11d0-a9b2-00c04fb6e6fc, version 1
     Endpoint: ncacn_np:\\XP[\PIPE\msgsvc]
     Annotation: Messenger Service


\end{verbatim}\end{itemize}
\subsubsection{Problems regarding : netbios-ssn (139/tcp)}
Security holes :\\
\begin{itemize}
\item \begin{verbatim}
. It was possible to log into the remote host using a NULL session.
The concept of a NULL session is to provide a null username and
a null password, which grants the user the 'guest' access

To prevent null sessions, see MS KB Article Q143474 (NT 4.0) and
Q246261 (Windows 2000). 
Note that this won't completely disable null sessions, but will 
prevent them from connecting to IPC$
Please see
 http://msgs.securepoint.com/cgi-bin/get/nessus-0204/50/1.html

. All the smb tests will be done as ''/'' in domain WORKGROUP      
CVE : CVE-2000-0222
BID : 990
\end{verbatim}\end{itemize}
Security warnings :\\
\begin{itemize}
\item \begin{verbatim}
The domain SID can be obtained remotely. Its value is :

WORKGROUP : 0-0-0-0-0

An attacker can use it to obtain the list of the local users of this
 host
Solution : filter the ports 137 to 139 and 445
Risk factor : Low

CVE : CVE-2000-1200
BID : 959
\end{verbatim}\item \begin{verbatim}
The host SID can be obtained remotely. Its value is :

XP : 5-21-583907252-2111687655-1957994488

An attacker can use it to obtain the list of the local users of this
 host
Solution : filter the ports 137 to 139 and 445
Risk factor : Low

CVE : CVE-2000-1200
BID : 959
\end{verbatim}\item \begin{verbatim}
Here is the browse list of the remote host : 

BENDER - 
XP - 


This is potentially dangerous as this may help the attack
of a potential hacker by giving him extra targets to check for

Solution : filter incoming traffic to this port
Risk factor : Low

\end{verbatim}\end{itemize}
Security note :\\
\begin{itemize}
\item \begin{verbatim}
The remote native lan manager is : Windows 2000 LAN Manager
The remote Operating System is : Windows 5.1
The remote SMB Domain Name is : WORKGROUP


\end{verbatim}\end{itemize}
\subsubsection{Problems regarding : blackjack (1025/tcp)}
Security note :\\
\begin{itemize}
\item \begin{verbatim}
A DCE service is listening on this port
     UUID: 1ff70682-0a51-30e8-076d-740be8cee98b, version 1
     Endpoint: ncacn_ip_tcp:10.163.155.6[1025]


\end{verbatim}\item \begin{verbatim}
A DCE service is listening on this port
     UUID: 378e52b0-c0a9-11cf-822d-00aa0051e40f, version 1
     Endpoint: ncacn_ip_tcp:10.163.155.6[1025]


\end{verbatim}\item \begin{verbatim}
A DCE service is listening on this port
     UUID: 0a74ef1c-41a4-4e06-83ae-dc74fb1cdd53, version 1
     Endpoint: ncacn_ip_tcp:10.163.155.6[1025]


\end{verbatim}\item \begin{verbatim}
A DCE service is listening on this port
     UUID: 5a7b91f8-ff00-11d0-a9b2-00c04fb6e6fc, version 1
     Endpoint: ncacn_ip_tcp:10.163.155.6[1025]
     Annotation: Messenger Service


\end{verbatim}\end{itemize}
\subsubsection{Problems regarding : general/tcp}
Security note :\\
\begin{itemize}
\item \begin{verbatim}
QueSO has found out that the remote host OS is 
* FreeBSD, NetBSD, OpenBSD  


CVE : CAN-1999-0454
\end{verbatim}\end{itemize}
\subsubsection{Problems regarding : general/udp}
Security note :\\
\begin{itemize}
\item \begin{verbatim}
For your information, here is the traceroute to 10.163.155.6 : 
?
10.163.155.6

\end{verbatim}\end{itemize}
\subsubsection{Problems regarding : netbios-ns (137/udp)}
Security warnings :\\
\begin{itemize}
\item \begin{verbatim}
. The following 7 NetBIOS names have been gathered :
 XP              = This is the computer name registered for
 workstation services by a WINS client.
 WORKGROUP       = Workgroup / Domain name
 XP              = Computer name that is registered for the messenger
 service on a computer that is a WINS client.
 XP             
 WORKGROUP       = Workgroup / Domain name (part of the Browser
 elections)
 WORKGROUP      
   __MSBROWSE__ 
. The remote host has the following MAC address on its adapter :
   0x00 0x60 0x1d 0x21 0xa9 0x49 

If you do not want to allow everyone to find the NetBios name
of your computer, you should filter incoming traffic to this port.

Risk factor : Medium
\end{verbatim}\end{itemize}
\subsubsection{Problems regarding : ms-term-serv (3389/tcp)}
Security note :\\
\begin{itemize}
\item \begin{verbatim}
The Terminal Services are enabled on the remote host.

Terminal Services allow a Windows user to remotely obtain
a graphical login (and therefore act as a local user on the
remote host).

If an attacker gains a valid login and password, he may
be able to use this service to gain further access
on the remote host.


Solution : Disable the Terminal Services if you do not use them
Risk factor : Low
\end{verbatim}\end{itemize}
\subsubsection{Problems regarding : unknown (1027/udp)}
Security note :\\
\begin{itemize}
\item \begin{verbatim}
A DCE service is listening on this port
     UUID: 5a7b91f8-ff00-11d0-a9b2-00c04fb6e6fc, version 1
     Endpoint: ncadg_ip_udp:10.163.155.6[1027]
     Annotation: Messenger Service


\end{verbatim}\end{itemize}
\newpage
\section{10.163.156.205}
\subsection{Open ports (TCP and UDP)}
\verb+10.163.156.205+ has the following ports that are open : 
\begin{itemize}
\item\verb+rtmp (1/tcp)+
\item\verb+telnet (23/tcp)+
\item\verb+ftp (21/tcp)+
\item\verb+chargen (19/tcp)+
\item\verb+daytime (13/tcp)+
\item\verb+discard (9/tcp)+
\item\verb+echo (7/tcp)+
\item\verb+smtp (25/tcp)+
\item\verb+time (37/tcp)+
\item\verb+finger (79/tcp)+
\item\verb+sunrpc (111/tcp)+
\item\verb+exec (512/tcp)+
\item\verb+printer (515/tcp)+
\item\verb+shell (514/tcp)+
\item\verb+login (513/tcp)+
\item\verb+ldaps (636/tcp)+
\item\verb+blackjack (1025/tcp)+
\item\verb+LSA-or-nterm (1026/tcp)+
\item\verb+kdm (1024/tcp)+
\item\verb+ms-lsa (1029/tcp)+
\item\verb+esl-lm (1455/tcp)+
\item\verb+general/tcp+
\item\verb+blackjack (1025/udp)+
\item\verb+sunrpc (111/udp)+
\item\verb+general/udp+
\item\verb+xdmcp (177/udp)+
\item\verb+echo (7/udp)+
\item\verb+daytime (13/udp)+
\end{itemize}
You should disable the services that you do not use, as they are potential security flaws.
\subsection{Details of the vulnerabilities}
\subsubsection{Problems regarding : rtmp (1/tcp)}
Security note :\\
\begin{itemize}
\item \begin{verbatim}
An unknown server is running on this port.
If you know what it is, please send this banner to the Nessus team:
00: 2d 53 65 72 76 69 63 65 20 6e 6f 74 20 61 76 61    -Service not
 ava
10: 69 6c 61 62 6c 65 0d 0a                            ilable..       
                 



\end{verbatim}\end{itemize}
\subsubsection{Problems regarding : telnet (23/tcp)}
Security holes :\\
\begin{itemize}
\item \begin{verbatim}
The account 'guest' has the password guest
An attacker may use it to gain further privileges on this system

Risk factor : High
Solution : Set a password for this account or disable it
CVE : CAN-1999-0502
\end{verbatim}\item \begin{verbatim}
The account 'demos' has no password set. 
An attacker may use it to gain further privileges on this system

Risk factor : High
Solution : Set a password for this account or disable it
CVE : CAN-1999-0502
\end{verbatim}\item \begin{verbatim}
The account 'EZsetup' has no password set. 
An attacker may use it to gain further privileges on this system

Risk factor : High
Solution : Set a password for this account or disable it
CVE : CAN-1999-0502
\end{verbatim}\item \begin{verbatim}
The account 'root' has the password root
An attacker may use it to gain further privileges on this system

Risk factor : High
Solution : Set a password for this account or disable it
CVE : CAN-1999-0502
\end{verbatim}\item \begin{verbatim}
The account 'lp' has no password set. 
An attacker may use it to gain further privileges on this system

Risk factor : High
Solution : Set a password for this account or disable it
CVE : CAN-1999-0502
\end{verbatim}\end{itemize}
Security warnings :\\
\begin{itemize}
\item \begin{verbatim}
The Telnet service is running.
This service is dangerous in the sense that
it is not ciphered - that is, everyone can sniff
the data that passes between the telnet client
and the telnet server. This includes logins
and passwords.

You should disable this service and use OpenSSH instead.
(www.openssh.com)

Solution : Comment out the 'telnet' line in /etc/inetd.conf.

Risk factor : Low
CVE : CAN-1999-0619
\end{verbatim}\end{itemize}
Security note :\\
\begin{itemize}
\item \begin{verbatim}
A telnet server seems to be running on this port
\end{verbatim}\item \begin{verbatim}
Remote telnet banner :


IRIX (IRIS)

\end{verbatim}\end{itemize}
\subsubsection{Problems regarding : ftp (21/tcp)}
Security note :\\
\begin{itemize}
\item \begin{verbatim}
An FTP server is running on this port.
Here is its banner : 
220 IRIS.fr.nessus.org FTP server ready.
\end{verbatim}\item \begin{verbatim}
Remote FTP server banner :
220 IRIS.fr.nessus.org FTP server ready.
\end{verbatim}\end{itemize}
\subsubsection{Problems regarding : chargen (19/tcp)}
Security warnings :\\
\begin{itemize}
\item \begin{verbatim}
The chargen service is running.
The 'chargen' service should only be enabled when testing the machine.
 

When contacted, chargen responds with some random characters
 (something
like all the characters in the alphabet in a row). When contacted via
 UDP, it 
will respond with a single UDP packet. When contacted via TCP, it will
 
continue spewing characters until the client closes the connection. 

An easy attack is 'pingpong' in which an attacker spoofs a packet
 between two
machines running chargen. This will cause them to spew characters at
 each 
other, slowing the machines down and saturating the network.
      
Solution : disable this service in /etc/inetd.conf.

Risk factor : Low
CVE : CVE-1999-0103
\end{verbatim}\end{itemize}
Security note :\\
\begin{itemize}
\item \begin{verbatim}
Chargen is running on this port
\end{verbatim}\end{itemize}
\subsubsection{Problems regarding : daytime (13/tcp)}
Security warnings :\\
\begin{itemize}
\item \begin{verbatim}
The daytime service is running.
The date format issued by this service
may sometimes help an attacker to guess
the operating system type. 

In addition to that, when the UDP version of
daytime is running, an attacker may link it 
to the echo port using spoofing, thus creating
a possible denial of service.

Solution : disable this service in /etc/inetd.conf.

Risk factor : Low
CVE : CVE-1999-0103
\end{verbatim}\end{itemize}
\subsubsection{Problems regarding : echo (7/tcp)}
Security warnings :\\
\begin{itemize}
\item \begin{verbatim}
The 'echo' port is open. This port is
not of any use nowadays, and may be a source of problems, 
since it can be used along with other ports to perform a denial
of service. You should really disable this service.

Risk factor : Low

Solution : comment out 'echo' in /etc/inetd.conf
CVE : CVE-1999-0103
\end{verbatim}\end{itemize}
Security note :\\
\begin{itemize}
\item \begin{verbatim}
An echo server is running on this port
\end{verbatim}\end{itemize}
\subsubsection{Problems regarding : smtp (25/tcp)}
Security holes :\\
\begin{itemize}
\item \begin{verbatim}
The remote sendmail server, according to its version number,
may be vulnerable to the -bt overflow attack which
allows any local user to execute arbitrary commands as root.

Solution : upgrade to the latest version of Sendmail
Risk factor : High
Note : This vulnerability is _local_ only
\end{verbatim}\item \begin{verbatim}
The remote sendmail server, according to its version number,
may be vulnerable to a buffer overflow its DNS handling code.

The owner of a malicious name server could use this flaw
to execute arbitrary code on this host.


Solution : Upgrade to Sendmail 8.12.5
Risk factor : High
CVE : CAN-2002-0906
BID : 5122
\end{verbatim}\end{itemize}
Security warnings :\\
\begin{itemize}
\item \begin{verbatim}
The remote SMTP server
answers to the EXPN and/or VRFY commands.

The EXPN command can be used to find 
the delivery address of mail aliases, or 
even the full name of the recipients, and 
the VRFY command may be used to check the 
validity of an account.


Your mailer should not allow remote users to
use any of these commands, because it gives
them too much information.


Solution : if you are using Sendmail, add the 
option
 O PrivacyOptions=goaway
in /etc/sendmail.cf.

Risk factor : Low
CVE : CAN-1999-0531
\end{verbatim}\item \begin{verbatim}
The remote sendmail server, according to its version number,
might be vulnerable to a queue destruction when a local user
runs
 sendmail -q -h1000

If you system does not allow users to process the queue (which
is the default), you are not vulnerable.

Solution : upgrade to the latest version of Sendmail or 
do not allow users to process the queue (RestrictQRun option)
Risk factor : Low
Note : This vulnerability is _local_ only
CVE : CAN-2001-0714
BID : 3378
\end{verbatim}\item \begin{verbatim}
According to the version number of the remote mail server, 
a local user may be able to obtain the complete mail configuration
and other interesting information about the mail queue even if
he is not allowed to access those information directly, by running
 sendmail -q -d0-nnnn.xxx
where nnnn & xxx are debugging levels.

If users are not allowed to process the queue (which is the default)
then you are not vulnerable.

Solution : upgrade to the latest version of Sendmail or 
do not allow users to process the queue (RestrictQRun option)
Risk factor : Very low / none
Note : This vulnerability is _local_ only
CVE : CAN-2001-0715
BID : 3898
\end{verbatim}\end{itemize}
Security note :\\
\begin{itemize}
\item \begin{verbatim}
An SMTP server is running on this port
Here is its banner : 
220 IRIS.fr.nessus.org ESMTP Sendmail SGI-8.9.3/8.9.3; Fri, 21 Feb
 2003 06:09:50 -0800 (PST)
\end{verbatim}\item \begin{verbatim}
Remote SMTP server banner :
220 IRIS.fr.nessus.org ESMTP Sendmail SGI-8.9.3/8.9.3; Fri, 21 Feb
 2003 06:11:07 -0800 (PST)

\end{verbatim}\item \begin{verbatim}
Nessus sent several emails containing the EICAR
test strings in them to the postmaster of
the remote SMTP server.

The EICAR test string is a fake virus which
triggers anti-viruses, in order to make sure
they run.

Nessus attempted to e-mail this string five times,
with different codings each time, in order to attempt
to fool the remote anti-virus (if any).


If there is an antivirus filter, these messages should
all be blocked.

*** To determine if the remote host is vulnerable, see
*** if any mail arrived to the postmaster of this host

Solution: Install an antivirus / upgrade it

Reference : http://online.securityfocus.com/archive/1/256619
Reference : http://online.securityfocus.com/archive/1/44301
Reference : http://online.securityfocus.com/links/188

Risk factor : Low
\end{verbatim}\end{itemize}
\subsubsection{Problems regarding : time (37/tcp)}
Security note :\\
\begin{itemize}
\item \begin{verbatim}
A time server seems to be running on this port
\end{verbatim}\end{itemize}
\subsubsection{Problems regarding : finger (79/tcp)}
Security warnings :\\
\begin{itemize}
\item \begin{verbatim}
The 'finger' service provides useful information
to attackers, since it allow them to gain usernames, check if a
 machine
is being used, and so on... 

Risk factor : Low

Solution : comment out the 'finger' line in /etc/inetd.conf
CVE : CVE-1999-0612
\end{verbatim}\end{itemize}
Security note :\\
\begin{itemize}
\item \begin{verbatim}
A finger server seems to be running on this port
\end{verbatim}\end{itemize}
\subsubsection{Problems regarding : sunrpc (111/tcp)}
Security note :\\
\begin{itemize}
\item \begin{verbatim}
RPC program #100000 version 2 'portmapper' (portmap sunrpc rpcbind) is
 running on this port
\end{verbatim}\end{itemize}
\subsubsection{Problems regarding : exec (512/tcp)}
Security warnings :\\
\begin{itemize}
\item \begin{verbatim}
The rexecd service is open. 
Because rexecd does not provide any good
means of authentication, it can be
used by an attacker to scan a third party
host, giving you troubles or bypassing
your firewall.

Solution : comment out the 'exec' line 
in /etc/inetd.conf.

Risk factor : Medium
CVE : CAN-1999-0618
\end{verbatim}\end{itemize}
\subsubsection{Problems regarding : printer (515/tcp)}
Security note :\\
\begin{itemize}
\item \begin{verbatim}
A LPD server seems to be running on this port
\end{verbatim}\end{itemize}
\subsubsection{Problems regarding : shell (514/tcp)}
Security warnings :\\
\begin{itemize}
\item \begin{verbatim}
The rsh service is running.
This service is dangerous in the sense that
it is not ciphered - that is, everyone can sniff
the data that passes between the rsh client
and the rsh server. This includes logins
and passwords.

You should disable this service and use ssh instead.

Solution : Comment out the 'rsh' line in /etc/inetd.conf.

Risk factor : Low
CVE : CAN-1999-0651
\end{verbatim}\end{itemize}
\subsubsection{Problems regarding : login (513/tcp)}
Security warnings :\\
\begin{itemize}
\item \begin{verbatim}
The rlogin service is running.
This service is dangerous in the sense that
it is not ciphered - that is, everyone can sniff
the data that passes between the rlogin client
and the rlogin server. This includes logins
and passwords.

You should disable this service and use openssh instead
(www.openssh.com)

Solution : Comment out the 'rlogin' line in /etc/inetd.conf.

Risk factor : Low
CVE : CAN-1999-0651
\end{verbatim}\end{itemize}
\subsubsection{Problems regarding : ldaps (636/tcp)}
Security note :\\
\begin{itemize}
\item \begin{verbatim}
RPC program #391017 version 1 is running on this port
\end{verbatim}\end{itemize}
\subsubsection{Problems regarding : blackjack (1025/tcp)}
Security note :\\
\begin{itemize}
\item \begin{verbatim}
RPC program #391029 version 1 is running on this port
\end{verbatim}\end{itemize}
\subsubsection{Problems regarding : LSA-or-nterm (1026/tcp)}
Security holes :\\
\begin{itemize}
\item \begin{verbatim}
The tooltalk RPC service is running.
An possible implementation fault in the 
ToolTalk object database server may allow an
attacker to execute arbitrary commands as
root.

*** This warning may be a false 
*** positive since the presence
*** of this vulnerability is only accurately
*** identified with local access.
    
Solution : Disable this service.
See also : CERT Advisory CA-98.11

Risk factor : High
CVE : CVE-1999-0003, CVE-1999-0693
BID : 122
\end{verbatim}\item \begin{verbatim}
The tooltalk RPC service is running.

There is a format string bug in many versions
of this service, which allow an attacker to gain
root remotely.

In addition to this, several versions of this service
allow remote attackers to overwrite abitrary memory
locations with a zero and possibly gain privileges
via a file descriptor argument in an AUTH_UNIX 
procedure call which is used as a table index by the
_TT_ISCLOSE procedure.

*** This warning may be a false positive since the presence
*** of the bug was not verified locally.
    
Solution : Disable this service or patch it
See also : CERT Advisories CA-2001-27 and CA-2002-20

Risk factor : High
CVE : CAN-2002-0677, CVE-2001-0717, CVE-2002-0679
BID : 3382
\end{verbatim}\end{itemize}
Security note :\\
\begin{itemize}
\item \begin{verbatim}
RPC program #100083 version 1 is running on this port
\end{verbatim}\end{itemize}
\subsubsection{Problems regarding : kdm (1024/tcp)}
Security warnings :\\
\begin{itemize}
\item \begin{verbatim}
The fam RPC service is running. 
Several versions of this service have
a well-known buffer overflow condition
that allows intruders to execute
arbitrary commands as root on this system.


Solution : disable this service in /etc/inetd.conf
More information :
 http://www.nai.com/nai_labs/asp_set/advisory/16_fam_adv.asp
Risk factor : High
CVE : CVE-1999-0059
BID : 353
\end{verbatim}\end{itemize}
Security note :\\
\begin{itemize}
\item \begin{verbatim}
RPC program #391002 version 1 'sgi_fam' (fam) is running on this port
\end{verbatim}\item \begin{verbatim}
RPC program #391002 version 2 'sgi_fam' (fam) is running on this port
\end{verbatim}\end{itemize}
\subsubsection{Problems regarding : esl-lm (1455/tcp)}
Security note :\\
\begin{itemize}
\item \begin{verbatim}
The service closed the connection after 0 seconds without sending any
 data
It might be protected by some TCP wrapper

\end{verbatim}\end{itemize}
\subsubsection{Problems regarding : general/tcp}
Security note :\\
\begin{itemize}
\item \begin{verbatim}
QueSO has found out that the remote host OS is 
* IRIX 6.x?  


CVE : CAN-1999-0454
\end{verbatim}\end{itemize}
\subsubsection{Problems regarding : blackjack (1025/udp)}
Security warnings :\\
\begin{itemize}
\item \begin{verbatim}
The rstatd RPC service is running. 
It provides an attacker interesting
information such as :

 - the CPU usage
 - the system uptime
 - its network usage
 - and more
 
Usually, it is not a good idea to let this
service open


Risk factor : Low
CVE : CAN-1999-0624
\end{verbatim}\end{itemize}
Security note :\\
\begin{itemize}
\item \begin{verbatim}
RPC program #100001 version 1 'rstatd' (rstat rup perfmeter rstat_svc)
 is running on this port
\end{verbatim}\item \begin{verbatim}
RPC program #100001 version 2 'rstatd' (rstat rup perfmeter rstat_svc)
 is running on this port
\end{verbatim}\item \begin{verbatim}
RPC program #100001 version 3 'rstatd' (rstat rup perfmeter rstat_svc)
 is running on this port
\end{verbatim}\end{itemize}
\subsubsection{Problems regarding : sunrpc (111/udp)}
Security note :\\
\begin{itemize}
\item \begin{verbatim}
RPC program #100000 version 2 'portmapper' (portmap sunrpc rpcbind) is
 running on this port
\end{verbatim}\end{itemize}
\subsubsection{Problems regarding : general/udp}
Security note :\\
\begin{itemize}
\item \begin{verbatim}
For your information, here is the traceroute to 10.163.156.205 : 
10.163.156.205

\end{verbatim}\end{itemize}
\subsubsection{Problems regarding : xdmcp (177/udp)}
Security warnings :\\
\begin{itemize}
\item \begin{verbatim}
The remote host is running XDMCP.

This protocol is used to provide X display connections for 
X terminals. XDMCP is completely insecure, since the traffic and
passwords are not encrypted. 

An attacker may use this flaw to capture all the keystrokes of
the users using this host through their X terminal, including
passwords.

Risk factor : Medium
Solution : Disable XDMCP
\end{verbatim}\end{itemize}
\subsubsection{Problems regarding : echo (7/udp)}
Security warnings :\\
\begin{itemize}
\item \begin{verbatim}
The 'echo' port is open. This port is
not of any use nowadays, and may be a source of problems, 
since it can be used along with other ports to perform a denial
of service. You should really disable this service.

Risk factor : Low

Solution : comment out 'echo' in /etc/inetd.conf
CVE : CVE-1999-0103
\end{verbatim}\end{itemize}
\subsubsection{Problems regarding : daytime (13/udp)}
Security warnings :\\
\begin{itemize}
\item \begin{verbatim}
The daytime service is running.
The date format issued by this service
may sometimes help an attacker to guess
the operating system type. 

In addition to that, when the UDP version of
daytime is running, an attacker may link it 
to the echo port using spoofing, thus creating
a possible denial of service.

Solution : disable this service in /etc/inetd.conf.

Risk factor : Low
CVE : CVE-1999-0103
\end{verbatim}\end{itemize}
\newpage
\section{10.163.156.16}
\subsection{Open ports (TCP and UDP)}
\verb+10.163.156.16+ has the following ports that are open : 
\begin{itemize}
\item\verb+smtp (25/tcp)+
\item\verb+telnet (23/tcp)+
\item\verb+ftp (21/tcp)+
\item\verb+chargen (19/tcp)+
\item\verb+daytime (13/tcp)+
\item\verb+discard (9/tcp)+
\item\verb+echo (7/tcp)+
\item\verb+time (37/tcp)+
\item\verb+finger (79/tcp)+
\item\verb+sunrpc (111/tcp)+
\item\verb+login (513/tcp)+
\item\verb+exec (512/tcp)+
\item\verb+printer (515/tcp)+
\item\verb+shell (514/tcp)+
\item\verb+uucp (540/tcp)+
\item\verb+xaudio (1103/tcp)+
\item\verb+general/tcp+
\item\verb+dtspc (6112/tcp)+
\item\verb+sunrpc (111/udp)+
\item\verb+sometimes-rpc8 (32772/udp)+
\item\verb+sometimes-rpc21 (32779/tcp)+
\item\verb+sometimes-rpc12 (32774/udp)+
\item\verb+sometimes-rpc14 (32775/udp)+
\item\verb+sometimes-rpc10 (32773/udp)+
\item\verb+unknown (32790/tcp)+
\item\verb+sometimes-rpc16 (32776/udp)+
\item\verb+unknown (32791/tcp)+
\item\verb+sometimes-rpc18 (32777/udp)+
\item\verb+sometimes-rpc20 (32778/udp)+
\item\verb+sometimes-rpc22 (32779/udp)+
\item\verb+lockd (4045/udp)+
\item\verb+unknown (32792/tcp)+
\item\verb+unknown (32793/tcp)+
\item\verb+sometimes-rpc24 (32780/udp)+
\item\verb+unknown (32794/tcp)+
\item\verb+lockd (4045/tcp)+
\item\verb+unknown (32812/udp)+
\item\verb+unknown (32795/tcp)+
\item\verb+unknown (32813/udp)+
\item\verb+unknown (32796/tcp)+
\item\verb+snmp (161/udp)+
\item\verb+xdmcp (177/udp)+
\item\verb+general/udp+
\item\verb+font-service (7100/tcp)+
\item\verb+echo (7/udp)+
\item\verb+daytime (13/udp)+
\end{itemize}
You should disable the services that you do not use, as they are potential security flaws.
\subsection{Details of the vulnerabilities}
\subsubsection{Problems regarding : smtp (25/tcp)}
Security holes :\\
\begin{itemize}
\item \begin{verbatim}
The remote SMTP server did not complain when issued the
command :
 MAIL FROM: |testing
 
This probably means that it is possible to send mail 
that will be bounced to a program, which is 
a serious threat, since this allows anyone to execute 
arbitrary commands on this host.

*** This security hole might be a false positive, since
*** some MTAs will not complain to this test, but instead
*** just drop the message silently
   
Solution : upgrade your MTA or change it.

Risk factor : High
CVE : CVE-1999-0203
BID : 2308
\end{verbatim}\end{itemize}
Security warnings :\\
\begin{itemize}
\item \begin{verbatim}
The remote SMTP server
answers to the EXPN and/or VRFY commands.

The EXPN command can be used to find 
the delivery address of mail aliases, or 
even the full name of the recipients, and 
the VRFY command may be used to check the 
validity of an account.


Your mailer should not allow remote users to
use any of these commands, because it gives
them too much information.


Solution : if you are using Sendmail, add the 
option
 O PrivacyOptions=goaway
in /etc/sendmail.cf.

Risk factor : Low
CVE : CAN-1999-0531
\end{verbatim}\item \begin{verbatim}
The remote SMTP server is vulnerable to a redirection
attack. That is, if a mail is sent to :

  user@hostname1@victim
  
Then the remote SMTP server (victim) will happily send the
mail to :
  user@hostname1
  
Using this flaw, an attacker may route a message
through your firewall, in order to exploit other
SMTP servers that can not be reached from the
outside.

*** THIS WARNING MAY BE A FALSE POSITIVE, SINCE
    SOME SMTP SERVERS LIKE POSTFIX WILL NOT
    COMPLAIN BUT DROP THIS MESSAGE ***
    
    
Solution : if you are using sendmail, then at the top
of ruleset 98, in /etc/sendmail.cf, insert :
R$*@$*@$*       $#error $@ 5.7.1 $: '551 Sorry, no redirections.'

Risk factor : Low
\end{verbatim}\item \begin{verbatim}
The remote SMTP server allows the relaying. This means that
it allows spammers to use your mail server to send their mails to
the world, thus wasting your network bandwidth.

Risk factor : Low/Medium

Solution : configure your SMTP server so that it can't be used as a
 relay
           any more.
CVE : CAN-1999-0512
\end{verbatim}\item \begin{verbatim}
The remote SMTP server allows anyone to
use it as a mail relay, provided that the source address 
is set to '<>'. 
This problem allows any spammer to use your mail server 
to spam the world, thus blacklisting your mailserver, and
using your network resources.

Risk factor : Medium

Solution : reconfigure this server properly
CVE : CVE-1999-0819
\end{verbatim}\item \begin{verbatim}
The remote SMTP server seems to allow remote users to
send mail anonymously by providing arguments that are 
too long to the HELO command (more than 1024 chars).

This problem may allow malicious users to send hate
mail or threatening mail using your server,
and keep their anonymity.

Risk factor : Low

Solution : If you are using sendmail, upgrade to
version 8.9.x or newer. If you do not run sendmail, contact
your vendor.
CVE : CAN-1999-0098
\end{verbatim}\end{itemize}
Security note :\\
\begin{itemize}
\item \begin{verbatim}
An unknown service is running on this port.
It is usually reserved for SMTP
\end{verbatim}\item \begin{verbatim}
Remote SMTP server banner :
220 unknown. Sendmail SMI-8.6/SMI-SVR4 ready at Fri, 21 Feb 2003
 15:10:24 GMT

\end{verbatim}\item \begin{verbatim}
An unknown server is running on this port.
If you know what it is, please send this banner to the Nessus team:
00: 32 32 30 20 75 6e 6b 6e 6f 77 6e 2e 20 53 65 6e    220 unknown.
 Sen
10: 64 6d 61 69 6c 20 53 4d 49 2d 38 2e 36 2f 53 4d    dmail
 SMI-8.6/SM
20: 49 2d 53 56 52 34 20 72 65 61 64 79 20 61 74 20    I-SVR4 ready at
 
30: 46 72 69 2c 20 32 31 20 46 65 62 20 32 30 30 33    Fri, 21 Feb
 2003
40: 20 31 35 3a 30 38 3a 33 38 20 47 4d 54 0d 0a 35     15:08:38
 GMT..5
50: 30 30 20 43 6f 6d 6d 61 6e 64 20 75 6e 72 65 63    00 Command
 unrec
60: 6f 67 6e 69 7a 65 64 0d 0a 35 30 30 20 43 6f 6d    ognized..500
 Com
70: 6d 61 6e 64 20 75 6e 72 65 63 6f 67 6e 69 7a 65    mand
 unrecognize
80: 64 0d 0a                                           d..            
                           



\end{verbatim}\end{itemize}
\subsubsection{Problems regarding : telnet (23/tcp)}
Security holes :\\
\begin{itemize}
\item \begin{verbatim}
The remote /bin/login seems to crash when it receives too many
environment variables.

An attacker may use this flaw to gain a root shell on this system.

See also : http://www.cert.org/advisories/CA-2001-34.html
Solution : Contact your vendor for a patch (or read the CERT advisory)
Risk factor : High
CVE : CVE-2001-0797
BID : 3681
\end{verbatim}\item \begin{verbatim}
The Telnet server does not return an expected number of replies
when it receives a long sequence of 'Are You There' commands.
This probably means it overflows one of its internal buffers and
crashes. It is likely an attacker could abuse this bug to gain
control over the remote host's superuser.

For more information, see:
http://www.team-teso.net/advisories/teso-advisory-011.tar.gz

Solution: Comment out the 'telnet' line in /etc/inetd.conf.
Risk factor : High
CVE : CVE-2001-0554
BID : 3064
\end{verbatim}\item \begin{verbatim}
There is a bug in the remote /bin/login which
allows an attacker to gain a shell on this host, without
even sending a shell code. 

An attacker may use this flaw to log in as any user
(except root) on the remote host.

Here is the output of the command 'cat /etc/passwd' :
cat /etc/passwd
root:x:0:1:Super-User:/:/sbin/sh
daemon:x:1:1::/:
bin:x:2:2::/usr/bin:
sys:x:3:3::/:
adm:x:4:4:Admin:/var/adm:
lp:x:71:8:Line Printer Admin:/usr/spool/lp:
smtp:x:0:0:Mail Daemon User:/:
uucp:x:5:5:uucp Admin:/usr/lib/uucp:
nuucp:x:9:9:uucp Admin:/var/spool/uucppublic:/usr/lib/uucp/uucico
listen:x:37:4:Network Admin:/usr/net/nls:
nobody:x:60001:60001:Nobody:/:
noaccess:x:60002:60002:No Access User:/:
nobody4:x:65534:65534:SunOS 4.x Nobody:/:
renaud:x:100:1::/home/renaud:/bin/sh
$ 

Solution : See http://www.cert.org/advisories/CA-2001-34.html
Risk factor : High
CVE : CVE-2001-0797
BID : 3681
\end{verbatim}\end{itemize}
Security warnings :\\
\begin{itemize}
\item \begin{verbatim}
The Telnet service is running.
This service is dangerous in the sense that
it is not ciphered - that is, everyone can sniff
the data that passes between the telnet client
and the telnet server. This includes logins
and passwords.

You should disable this service and use OpenSSH instead.
(www.openssh.com)

Solution : Comment out the 'telnet' line in /etc/inetd.conf.

Risk factor : Low
CVE : CAN-1999-0619
\end{verbatim}\end{itemize}
Security note :\\
\begin{itemize}
\item \begin{verbatim}
A telnet server seems to be running on this port
\end{verbatim}\item \begin{verbatim}
Remote telnet banner :


SunOS 5.6

\end{verbatim}\end{itemize}
\subsubsection{Problems regarding : ftp (21/tcp)}
Security holes :\\
\begin{itemize}
\item \begin{verbatim}
You seem to be running an FTP server which is vulnerable to the
'glob heap corruption' flaw.
An attacker may use this problem to execute arbitrary commands on this
 host.

*** Nessus relied solely on the banner of the server to issue this
 warning,
*** so this alert might be a false positive

Solution : Upgrade your ftp server software to the latest version.
Risk factor : High

CVE : CVE-2001-0550
BID : 3581
\end{verbatim}\end{itemize}
Security note :\\
\begin{itemize}
\item \begin{verbatim}
An FTP server is running on this port.
Here is its banner : 
220 unknown FTP server (SunOS 5.6) ready.
\end{verbatim}\item \begin{verbatim}
Remote FTP server banner :
220 unknown FTP server (SunOS 5.6) ready.
\end{verbatim}\end{itemize}
\subsubsection{Problems regarding : chargen (19/tcp)}
Security warnings :\\
\begin{itemize}
\item \begin{verbatim}
The chargen service is running.
The 'chargen' service should only be enabled when testing the machine.
 

When contacted, chargen responds with some random characters
 (something
like all the characters in the alphabet in a row). When contacted via
 UDP, it 
will respond with a single UDP packet. When contacted via TCP, it will
 
continue spewing characters until the client closes the connection. 

An easy attack is 'pingpong' in which an attacker spoofs a packet
 between two
machines running chargen. This will cause them to spew characters at
 each 
other, slowing the machines down and saturating the network.
      
Solution : disable this service in /etc/inetd.conf.

Risk factor : Low
CVE : CVE-1999-0103
\end{verbatim}\end{itemize}
Security note :\\
\begin{itemize}
\item \begin{verbatim}
Chargen is running on this port
\end{verbatim}\end{itemize}
\subsubsection{Problems regarding : daytime (13/tcp)}
Security warnings :\\
\begin{itemize}
\item \begin{verbatim}
The daytime service is running.
The date format issued by this service
may sometimes help an attacker to guess
the operating system type. 

In addition to that, when the UDP version of
daytime is running, an attacker may link it 
to the echo port using spoofing, thus creating
a possible denial of service.

Solution : disable this service in /etc/inetd.conf.

Risk factor : Low
CVE : CVE-1999-0103
\end{verbatim}\end{itemize}
\subsubsection{Problems regarding : echo (7/tcp)}
Security warnings :\\
\begin{itemize}
\item \begin{verbatim}
The 'echo' port is open. This port is
not of any use nowadays, and may be a source of problems, 
since it can be used along with other ports to perform a denial
of service. You should really disable this service.

Risk factor : Low

Solution : comment out 'echo' in /etc/inetd.conf
CVE : CVE-1999-0103
\end{verbatim}\end{itemize}
Security note :\\
\begin{itemize}
\item \begin{verbatim}
An echo server is running on this port
\end{verbatim}\end{itemize}
\subsubsection{Problems regarding : time (37/tcp)}
Security note :\\
\begin{itemize}
\item \begin{verbatim}
A time server seems to be running on this port
\end{verbatim}\end{itemize}
\subsubsection{Problems regarding : finger (79/tcp)}
Security warnings :\\
\begin{itemize}
\item \begin{verbatim}
The 'finger' service provides useful information
to attackers, since it allow them to gain usernames, check if a
 machine
is being used, and so on... 

Risk factor : Low

Solution : comment out the 'finger' line in /etc/inetd.conf
CVE : CVE-1999-0612
\end{verbatim}\item \begin{verbatim}
The remote finger daemon accepts
to redirect requests. That is, users can perform
requests like :
  finger user@host@victim

This allows an attacker to use your computer
as a relay to gather information on another
network, making the other network think you
are making the requests.

Solution: disable your finger daemon (comment out
the finger line in /etc/inetd.conf) or 
install a more secure one.

Risk factor : Low
CVE : CAN-1999-0105
\end{verbatim}\item \begin{verbatim}
There is a bug in the finger service
which will make it display the list of the accounts that
have never been used, when anyone issues the request :

  finger 'a b c d e f g h'@target
  
This list will help an attacker to guess the operating
system type. It will also tell him which accounts have
never been used, which will often make him focus his
attacks on these accounts.

Solution : disable the finger service in /etc/inetd.conf, or
apply the patches from Sun.

Risk factor : Medium
BID : 3457
\end{verbatim}\end{itemize}
Security note :\\
\begin{itemize}
\item \begin{verbatim}
A finger server seems to be running on this port
\end{verbatim}\end{itemize}
\subsubsection{Problems regarding : sunrpc (111/tcp)}
Security note :\\
\begin{itemize}
\item \begin{verbatim}
RPC program #100000 version 4 'portmapper' (portmap sunrpc rpcbind) is
 running on this port
\end{verbatim}\item \begin{verbatim}
RPC program #100000 version 3 'portmapper' (portmap sunrpc rpcbind) is
 running on this port
\end{verbatim}\item \begin{verbatim}
RPC program #100000 version 2 'portmapper' (portmap sunrpc rpcbind) is
 running on this port
\end{verbatim}\end{itemize}
\subsubsection{Problems regarding : login (513/tcp)}
Security warnings :\\
\begin{itemize}
\item \begin{verbatim}
The rlogin service is running.
This service is dangerous in the sense that
it is not ciphered - that is, everyone can sniff
the data that passes between the rlogin client
and the rlogin server. This includes logins
and passwords.

You should disable this service and use openssh instead
(www.openssh.com)

Solution : Comment out the 'rlogin' line in /etc/inetd.conf.

Risk factor : Low
CVE : CAN-1999-0651
\end{verbatim}\end{itemize}
\subsubsection{Problems regarding : exec (512/tcp)}
Security warnings :\\
\begin{itemize}
\item \begin{verbatim}
The rexecd service is open. 
Because rexecd does not provide any good
means of authentication, it can be
used by an attacker to scan a third party
host, giving you troubles or bypassing
your firewall.

Solution : comment out the 'exec' line 
in /etc/inetd.conf.

Risk factor : Medium
CVE : CAN-1999-0618
\end{verbatim}\end{itemize}
\subsubsection{Problems regarding : printer (515/tcp)}
Security note :\\
\begin{itemize}
\item \begin{verbatim}
A LPD server seems to be running on this port
\end{verbatim}\end{itemize}
\subsubsection{Problems regarding : shell (514/tcp)}
Security warnings :\\
\begin{itemize}
\item \begin{verbatim}
The rsh service is running.
This service is dangerous in the sense that
it is not ciphered - that is, everyone can sniff
the data that passes between the rsh client
and the rsh server. This includes logins
and passwords.

You should disable this service and use ssh instead.

Solution : Comment out the 'rsh' line in /etc/inetd.conf.

Risk factor : Low
CVE : CAN-1999-0651
\end{verbatim}\end{itemize}
\subsubsection{Problems regarding : uucp (540/tcp)}
Security note :\\
\begin{itemize}
\item \begin{verbatim}
The service closed the connection after 0 seconds without sending any
 data
It might be protected by some TCP wrapper

\end{verbatim}\end{itemize}
\subsubsection{Problems regarding : general/tcp}
Security note :\\
\begin{itemize}
\item \begin{verbatim}
QueSO has found out that the remote host OS is 
* Solaris 2.x


CVE : CAN-1999-0454
\end{verbatim}\end{itemize}
\subsubsection{Problems regarding : dtspc (6112/tcp)}
Security holes :\\
\begin{itemize}
\item \begin{verbatim}
The 'dtspcd' service is running.

Some versions of this daemon are vulnerable to
a buffer overflow attack which allows an attacker
to gain root privileges

*** This warning might be a false positive,
*** as no real overflow was performed

Solution : See http://www.cert.org/advisories/CA-2001-31.html
to determine if you are vulnerable or deactivate
this service (comment out the line 'dtspc' in /etc/inetd.conf)

Risk factor : High
CVE : CVE-2001-0803
BID : 3517
\end{verbatim}\end{itemize}
\subsubsection{Problems regarding : sunrpc (111/udp)}
Security note :\\
\begin{itemize}
\item \begin{verbatim}
RPC program #100000 version 4 'portmapper' (portmap sunrpc rpcbind) is
 running on this port
\end{verbatim}\item \begin{verbatim}
RPC program #100000 version 3 'portmapper' (portmap sunrpc rpcbind) is
 running on this port
\end{verbatim}\item \begin{verbatim}
RPC program #100000 version 2 'portmapper' (portmap sunrpc rpcbind) is
 running on this port
\end{verbatim}\end{itemize}
\subsubsection{Problems regarding : sometimes-rpc8 (32772/udp)}
Security note :\\
\begin{itemize}
\item \begin{verbatim}
RPC program #100300 version 3 'nisd' (rpc.nisd) is running on this
 port
\end{verbatim}\end{itemize}
\subsubsection{Problems regarding : sometimes-rpc21 (32779/tcp)}
Security note :\\
\begin{itemize}
\item \begin{verbatim}
RPC program #100300 version 3 'nisd' (rpc.nisd) is running on this
 port
\end{verbatim}\end{itemize}
\subsubsection{Problems regarding : sometimes-rpc12 (32774/udp)}
Security note :\\
\begin{itemize}
\item \begin{verbatim}
RPC program #100232 version 10 'sadmind' is running on this port
\end{verbatim}\end{itemize}
\subsubsection{Problems regarding : sometimes-rpc14 (32775/udp)}
Security note :\\
\begin{itemize}
\item \begin{verbatim}
RPC program #100011 version 1 'rquotad' (rquotaprog quota rquota) is
 running on this port
\end{verbatim}\end{itemize}
\subsubsection{Problems regarding : sometimes-rpc10 (32773/udp)}
Security note :\\
\begin{itemize}
\item \begin{verbatim}
RPC program #100024 version 1 'status' is running on this port
\end{verbatim}\end{itemize}
\subsubsection{Problems regarding : unknown (32790/tcp)}
Security note :\\
\begin{itemize}
\item \begin{verbatim}
RPC program #100024 version 1 'status' is running on this port
\end{verbatim}\end{itemize}
\subsubsection{Problems regarding : sometimes-rpc16 (32776/udp)}
Security note :\\
\begin{itemize}
\item \begin{verbatim}
RPC program #100002 version 2 'rusersd' (rusers) is running on this
 port
\end{verbatim}\item \begin{verbatim}
RPC program #100002 version 3 'rusersd' (rusers) is running on this
 port
\end{verbatim}\end{itemize}
\subsubsection{Problems regarding : unknown (32791/tcp)}
Security note :\\
\begin{itemize}
\item \begin{verbatim}
RPC program #100002 version 2 'rusersd' (rusers) is running on this
 port
\end{verbatim}\item \begin{verbatim}
RPC program #100002 version 3 'rusersd' (rusers) is running on this
 port
\end{verbatim}\end{itemize}
\subsubsection{Problems regarding : sometimes-rpc18 (32777/udp)}
Security note :\\
\begin{itemize}
\item \begin{verbatim}
RPC program #100012 version 1 'sprayd' (spray) is running on this port
\end{verbatim}\end{itemize}
\subsubsection{Problems regarding : sometimes-rpc20 (32778/udp)}
Security note :\\
\begin{itemize}
\item \begin{verbatim}
RPC program #100008 version 1 'walld' (rwall shutdown) is running on
 this port
\end{verbatim}\end{itemize}
\subsubsection{Problems regarding : sometimes-rpc22 (32779/udp)}
Security note :\\
\begin{itemize}
\item \begin{verbatim}
RPC program #100001 version 2 'rstatd' (rstat rup perfmeter rstat_svc)
 is running on this port
\end{verbatim}\item \begin{verbatim}
RPC program #100001 version 3 'rstatd' (rstat rup perfmeter rstat_svc)
 is running on this port
\end{verbatim}\item \begin{verbatim}
RPC program #100001 version 4 'rstatd' (rstat rup perfmeter rstat_svc)
 is running on this port
\end{verbatim}\end{itemize}
\subsubsection{Problems regarding : lockd (4045/udp)}
Security note :\\
\begin{itemize}
\item \begin{verbatim}
RPC program #100021 version 1 'nlockmgr' is running on this port
\end{verbatim}\item \begin{verbatim}
RPC program #100021 version 2 'nlockmgr' is running on this port
\end{verbatim}\item \begin{verbatim}
RPC program #100021 version 3 'nlockmgr' is running on this port
\end{verbatim}\item \begin{verbatim}
RPC program #100021 version 4 'nlockmgr' is running on this port
\end{verbatim}\end{itemize}
\subsubsection{Problems regarding : unknown (32792/tcp)}
Security note :\\
\begin{itemize}
\item \begin{verbatim}
RPC program #100221 version 1 is running on this port
\end{verbatim}\end{itemize}
\subsubsection{Problems regarding : unknown (32793/tcp)}
Security note :\\
\begin{itemize}
\item \begin{verbatim}
RPC program #100235 version 1 is running on this port
\end{verbatim}\end{itemize}
\subsubsection{Problems regarding : sometimes-rpc24 (32780/udp)}
Security note :\\
\begin{itemize}
\item \begin{verbatim}
RPC program #100068 version 2 is running on this port
\end{verbatim}\item \begin{verbatim}
RPC program #100068 version 3 is running on this port
\end{verbatim}\item \begin{verbatim}
RPC program #100068 version 4 is running on this port
\end{verbatim}\item \begin{verbatim}
RPC program #100068 version 5 is running on this port
\end{verbatim}\end{itemize}
\subsubsection{Problems regarding : unknown (32794/tcp)}
Security note :\\
\begin{itemize}
\item \begin{verbatim}
RPC program #100083 version 1 is running on this port
\end{verbatim}\end{itemize}
\subsubsection{Problems regarding : lockd (4045/tcp)}
Security note :\\
\begin{itemize}
\item \begin{verbatim}
RPC program #100021 version 1 'nlockmgr' is running on this port
\end{verbatim}\item \begin{verbatim}
RPC program #100021 version 2 'nlockmgr' is running on this port
\end{verbatim}\item \begin{verbatim}
RPC program #100021 version 3 'nlockmgr' is running on this port
\end{verbatim}\item \begin{verbatim}
RPC program #100021 version 4 'nlockmgr' is running on this port
\end{verbatim}\end{itemize}
\subsubsection{Problems regarding : unknown (32812/udp)}
Security note :\\
\begin{itemize}
\item \begin{verbatim}
RPC program #300598 version 1 is running on this port
\end{verbatim}\item \begin{verbatim}
RPC program #805306368 version 1 is running on this port
\end{verbatim}\end{itemize}
\subsubsection{Problems regarding : unknown (32795/tcp)}
Security note :\\
\begin{itemize}
\item \begin{verbatim}
RPC program #300598 version 1 is running on this port
\end{verbatim}\item \begin{verbatim}
RPC program #805306368 version 1 is running on this port
\end{verbatim}\end{itemize}
\subsubsection{Problems regarding : unknown (32813/udp)}
Security note :\\
\begin{itemize}
\item \begin{verbatim}
RPC program #100249 version 1 is running on this port
\end{verbatim}\end{itemize}
\subsubsection{Problems regarding : unknown (32796/tcp)}
Security note :\\
\begin{itemize}
\item \begin{verbatim}
RPC program #100249 version 1 is running on this port
\end{verbatim}\end{itemize}
\subsubsection{Problems regarding : snmp (161/udp)}
Security holes :\\
\begin{itemize}
\item \begin{verbatim}
SNMP Agent responded as expected with community name: public
SNMP Agent responded as expected with community name: private
SNMP Agent responded as expected with community name: all private
CVE : CAN-1999-0186
BID : 177
\end{verbatim}\end{itemize}
Security warnings :\\
\begin{itemize}
\item \begin{verbatim}
It was possible to obtain the list of network interfaces of the
remote host via SNMP : 

. /etc/snmp/conf/snmpdx.rsrc
. /etc/snmp/conf

An attacker may use this information to gain more knowledge about
the target host.
Solution : disable the SNMP service on the remote host if you do not
           use it, or filter incoming UDP packets going to this port
Risk factor : Low
\end{verbatim}\end{itemize}
\subsubsection{Problems regarding : xdmcp (177/udp)}
Security warnings :\\
\begin{itemize}
\item \begin{verbatim}
The remote host is running XDMCP.

This protocol is used to provide X display connections for 
X terminals. XDMCP is completely insecure, since the traffic and
passwords are not encrypted. 

An attacker may use this flaw to capture all the keystrokes of
the users using this host through their X terminal, including
passwords.

Risk factor : Medium
Solution : Disable XDMCP
\end{verbatim}\end{itemize}
\subsubsection{Problems regarding : general/udp}
Security note :\\
\begin{itemize}
\item \begin{verbatim}
For your information, here is the traceroute to 10.163.156.16 : 
?
10.163.156.16

\end{verbatim}\end{itemize}
\subsubsection{Problems regarding : font-service (7100/tcp)}
Security holes :\\
\begin{itemize}
\item \begin{verbatim}
The remote X Font Service (xfs) might be vulnerable to a buffer
overflow.

An attacker may use this flaw to gain root on this host
remotely.

*** Note that Nessus did not actually check for the flaw
*** as details about this vulnerability are still unknown

Solution : See CERT Advisory CA-2002-34
Risk factor : High
CVE : CAN-2002-1317
\end{verbatim}\end{itemize}
\subsubsection{Problems regarding : echo (7/udp)}
Security warnings :\\
\begin{itemize}
\item \begin{verbatim}
The 'echo' port is open. This port is
not of any use nowadays, and may be a source of problems, 
since it can be used along with other ports to perform a denial
of service. You should really disable this service.

Risk factor : Low

Solution : comment out 'echo' in /etc/inetd.conf
CVE : CVE-1999-0103
\end{verbatim}\end{itemize}
\subsubsection{Problems regarding : daytime (13/udp)}
Security warnings :\\
\begin{itemize}
\item \begin{verbatim}
The daytime service is running.
The date format issued by this service
may sometimes help an attacker to guess
the operating system type. 

In addition to that, when the UDP version of
daytime is running, an attacker may link it 
to the echo port using spoofing, thus creating
a possible denial of service.

Solution : disable this service in /etc/inetd.conf.

Risk factor : Low
CVE : CVE-1999-0103
\end{verbatim}\end{itemize}
\newpage
\section*{Conclusion}
A security scanner, such as Nessus, is not a garantee  of the security of your network.\\
A lot of factors can not be tested by a security scanner : the practices of the users of the network, the home-made services and CGIs, and so on... So, you should not have a false sense of security now that the test are done. We recommand that you monitor actively what happens on your firewall, and that you use some tools such as tripwire to restore your servers more easily in the case of an intrusion.\\
In addition to that, you must know that new security holes are found each week. That is why we recommand that you visit \verb+http://www.nessus.org/scripts.html+, which is a page that contains the test for all the holes that are published on public mailing lists such as BugTraq (see \verb+http://www.securityfocus.com+ for details) and test the security of your network on a (at least) weekly basis with the checks that are on this page.\\
\textit{This report was generated with Nessus, the open-sourced security scanner. See http://www.nessus.org for more information}\end{document}
